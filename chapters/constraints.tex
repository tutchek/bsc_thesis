\chapter{List of the constraints}
\label{constraints}

In this appendix we provide a list of supported constraints foe each of studied 
constraint solvers. The constraints of the same type have often different names
in the system. Therefore we use a symbolic expression which describes the constraint.
The following convention is used: the integer or real variables are typed as 
text in the latin small alphabet, $x$, $y$, $a$ for example. 
Boolean variables are denoted by the monospace font, for example \texttt{b}, \texttt{x}. 
If more variables have  The constants which are not to be 
constrained are typed as the small greek letters, for example $\alpha$, $\epsilon$. The set
variables are denoted by the capital latin letters, for example $S$, $X$, $P$. Vector
of variables is denoted by the square brackets. For example a vector of the set variables would
be typed $[X,Y,S]$. The subscript index after the ending bracket of the vector means specification
of the element of the vector. To denote second item of vector $[x,y,x]$ we write $[x,y,z]_2$.
All standard mathematical symbols do have their standard meaning. 
If the constraint cannot be described by the mathematical expression the standard
name is used. The definition of such constraint follows at the end of the appropriate section.  

The symbol \ano means that the constraint is fully supported in the constraint solver.
The symbol \trochu means partial support of the constraint. The explanation follows under
the table. The symbol \vubec means that this constraint is not supported by the standard
version of the solver. We use following abbreviations for the constraint solver names:
Moz means Mozart/Oz, Cho means Choco, Min means Minion, ECL means \eclipse and finally
SiP means SICStus Prolog.

\section{Domain constraints}
\begin{center}
\begin{tabular}{|l||c|c|c|c|c|c|}
\hline Constraint & Moz & Cho & Min & Gec & ECL & SiP \\
\hline 
\hline $x = n$ & ? & ? & ? & \ano & ? & ? \\
\hline $x \in \{\alpha, ..., \beta\}$ & ? & ? & ? & \ano & ? & ? \\
\hline $x \in S$ & ? & ? & ? & \ano & ? & ? \\
\hline $(x = n) \Leftrightarrow \mathtt{b}$ & ? & ? & ? & \ano & ? & ? \\
\hline $(x \in \{\alpha ... \beta\}) \Leftrightarrow \mathtt{b}$ & ? & ? & ? & \ano & ? & ? \\
\hline $(x \in S) \Leftrightarrow \mathtt{b}$ & ? & ? & ? & \ano & ? & ? \\
\hline 
\end{tabular}
\end{center}

\section{Relation constraints}
\begin{center}
\begin{tabular}{|l||c|c|c|c|c|c|}
\hline Constraint & Moz & Cho & Min & Gec & ECL & SiP \\
\hline 

\hline $x > y$ & ? & ? & ? & \ano & ? & ? \\
\hline $x \geq y$ & ? & ? & ? & \ano & ? & ? \\
\hline $x = y$ & ? & ? & ? & \ano & ? & ? \\
\hline $x \leq y$ & ? & ? & ? & \ano & ? & ? \\
\hline $x < y$ & ? & ? & ? & \ano & ? & ? \\
\hline $x \neq y$ & ? & ? & ? & \ano & ? & ? \\

\hline $\sum{[x_0,...,x_\alpha]} > y$      & ? & ? & \vubec & \ano & ? & ? \\
\hline $\sum{[x_0,...,x_\alpha]} \geq y$   & ? & ? & \ano & \ano & ? & ? \\
\hline $\sum{[x_0,...,x_\alpha]} = y$      & ? & ? & \vubec & \ano & ? & ? \\
\hline $\sum{[x_0,...,x_\alpha]} \leq y$   & ? & ? & \ano & \ano & ? & ? \\
\hline $\sum{[x_0,...,x_\alpha]} < y$      & ? & ? & \vubec & \ano & ? & ? \\
\hline $\sum{[x_0,...,x_\alpha]} \neq y$   & ? & ? & \vubec & \ano & ? & ? \\

\hline $\bigwedge {\mathtt{x}_i} \wedge \bigwedge{\neg \mathtt{y}_i} = \mathtt{b}$   & ? & ? & ? & \ano & ? & ? \\
\hline $\bigvee {\mathtt{x}_i} \vee \bigvee{\neg \mathtt{y}_i} = \mathtt{b}$   & ? & ? & ? & \ano & ? & ? \\
\hline 
\end{tabular}
\end{center}

\section{Element constraints}
\begin{center}
\begin{tabular}{|l||c|c|c|c|c|c|}
\hline Constraint & Moz & Cho & Min & Gec & ECL & SiP \\
\hline 

\hline $[\alpha_1, ..., \alpha_\beta]_x = \gamma$ & ? & ? & ? & \ano & ? & ? \\
\hline $[\alpha_1, ..., \alpha_\beta]_x = y$ & ? & ? & ? & \ano & ? & ? \\
\hline $[x_1, ..., x_\alpha]_y = \beta$ & ? & ? & ? & \ano & ? & ? \\
\hline $[x_1, ..., x_\alpha]_y = z$ & ? & ? & ? & \ano & ? & ? \\

\hline 
\end{tabular}
\end{center}

\section{Global constraints}
\begin{center}
\begin{tabular}{|l||c|c|c|c|c|c|}
\hline Constraint & Moz & Cho & Min & Gec & ECL & SiP \\
\hline 

\hline $\mathrm{alldifferent}([x_1, ..., x_\alpha])$ & \ano & \ano & \ano & \ano & \ano & \ano \\
\hline $[x_1, ..., x_\alpha]_i = j \Leftrightarrow [y_1, ..., y_\beta]_j = i$ & ? & ? & ? & \ano & ? & ? \\

\hline 
\end{tabular}
\end{center}

The alldifferent constraint ensures that all values in a given vector are distinct.