\chapter{List of the constraints}
\label{constraints}

In this appendix we present the comparison of modelling capabilites of the solvers 
based on the constraints which are supported. We used the manuals of the systems \cite{choco:documentation, mozart:documentation, gecode:reference, sicstus:manual, eclipse:cspmanual, eclipse:tutorial, minion:manual, gecode:modelling} as
the source of the data. We do not want to provide a full
list of the supported constraints. Our goal is
to show the most important constraints or the constraints which defines the differences 
between the solvers. For each group of the constraint we present a comparison table. The used symbols have the following meaning:
The constraint variables are denoted by the emphatised small latin letters ($x$, $y$, $a$).
The constants are symbolised by the greek small letters ($\alpha$, $\beta$, $\gamma$).
The vectors of the constraint variables or the constants are in bold ($\mathbf{x}$, $\mathbf{a}$, $\bm{\alpha}$).
As usual in programming languages by square brackets applied on the vector we 
denote the element of a vector ($\mathbf{x}[y]$). Finally the boolean variables are denoted by
the monospace font ($\mathtt{x}, \mathtt{b}$). The symbol tilde ($\sim$) means a relation
operator (any of the $=$, $\neq$, $>$, $<$, $\leq$, $\geq$). The other symbols have their usual meaning.
If it is not possible to describe the constraint with the mathematical expression
we use a symbolic name instead. If the solver fully supports the constraint the \ano symbol
is used. If the constraint is supported only partial the symbol \trochu is used. 
If the solver does not support the constraint at all we use the symbol \vubec. 
We say that the solver supports a constraint only if the constraints is available
as a single constraint which can be directly used or it there exists an expression 
similar to the constraint. In the other case even if the constraint effect can be achieved
by the combination of some other constraints we consider that constraint as unsupported. 
We use the following abbreviations for the constraint solvers:
Moz means Mozart/Oz, Cho means Choco, Min means Minion, ECL means \eclipse and finally
SiP means SICStus Prolog.

\begin{table}
\caption{\label{constraints:relation}Relation and arithmetic constraints}
\begin{center}
\begin{tabular}{|p{5cm}||c|c|c|c|c|c|}
\hline Constraint & Moz & Cho & Min & Gec & ECL & SiP \\
\hline 

\hline $x \sim y$ 																												& \ano & \ano & \trochu & \ano & \ano & \ano \\
\hline $\sum_i{\mathbf{x}_i} \sim y$   																		& \ano & \ano & \trochu & \ano & \ano & \ano \\
\hline $\sum_i{\left(\mathbf{x}_i\mathbf{w}_i\right)} \sim y$      				& \ano & \ano & \trochu & \ano & \vubec & \ano \\
\hline $\sum_i{\left(\mathbf{x}_i\prod_j\mathbf{w}_{ij}\right)} \sim y$   & \ano & \vubec & \vubec & \vubec & \vubec & \vubec \\

\hline $\bigwedge_i {\mathtt{x}_i} \wedge \bigwedge_i{\neg \mathtt{y}_i} = \mathtt{b}$   & \vubec & \vubec & \vubec & \ano & \vubec & \vubec \\
\hline $\bigvee_i {\mathtt{x}_i} \vee \bigvee_i{\neg \mathtt{y}_i} = \mathtt{b}$   			 & \vubec & \vubec & \vubec & \ano & \vubec & \vubec \\

\hline $|x| = y$ 										& \ano & \ano & \ano & \ano & \ano & \ano \\
\hline $|x-y| \sim z$ 							& \ano & \vubec & \trochu & \vubec & \vubec & \ano \\
\hline $x \, \mathrm{mod} \, y = z$ & \ano & \ano & \ano & \ano & \ano & \ano \\

\hline 
\end{tabular}
\end{center}
\end{table}

We divide the constraints into four groups -- {\em relation and arithmetic}, {\em global}, {\em scheduling} and {\em other} constraints.
The most of the relation and arithmetic constraints (see the table \ref{constraints:relation}) are 
supported in the all solvers. As we can see the Minion system has for some of the
constraints only the partial support. For the sums it means that Minion supports
only the $\leq$ and $\geq$ operators. The partialy supported constraint $x \sim y$ on the other hand
means that it supports only the $=$ and $\neq$ operators and finally for the $|x-y| \sim z$ constraint
only the operator $=$ is supported. The global constraints (see the table \ref{constraints:global})
are the constraints which constrain the vectors of variables. The basic global constraint
is {\em alldifferent} which constrains the given vector $\mathbf{x}$ such that each
variable contained in the vector has a different value. The {\em hamming} constraints
ensures that the hamming distance of the two given vectors is at least $c$.
The scheduling constraints as the name suggest offers the support for modelling
of the scheduling problems. The {\em cummulative} constraint is used if we need to
schedule the tasks on the given machines. The {\em overlaps} and the {\em disjunctive}
are used to constrain the relative order of the taks. Finally in the fourth group
there are the other constraints. By using the {\em extensional} constraint we can 
constrain the complex relations of the variables by specifying the set of compatible 
values. The constraint then constrain the variables such that they have only the 
compatible values. The {DFA} stands for the deterministic fiinite automaton. The
constraint enforces that the given vector of variables is the word which is accepted
by the solver. Finally the {hamiltonian circle} ensures that the given vector of variables 
is a hamiltonian circle in the graph. In the \eclipse we can define the extensional 
constraint by using of the Propia library. We define the predicate with the compatible
values and the use the Propia library tools on that predicate.  



\begin{table}
\caption{\label{constraints:global}Global constraints}
\begin{center}
\begin{tabular}{|p{5cm}||c|c|c|c|c|c|}
\hline Constraint & Moz & Cho & Min & Gec & ECL & SiP \\
\hline 

\hline $\mathrm{alldifferent}(\mathbf{x})$ 										& \ano & \ano & \ano & \ano & \ano & \ano \\
\hline $\mathbf{x}[i] = j \Leftrightarrow \mathbf{y}[j] = i$ 	& \vubec & \ano & \vubec & \ano & \vubec & \ano \\
\hline $\mathrm{hamming}(\mathbf{x},\mathbf{y}) \geq c$ 			& \vubec & \vubec & \ano & \vubec & \vubec & \vubec \\
\hline $\mathbf{x}[y] = z$ 																		& \ano & \ano & \ano & \ano & \ano & \ano \\

\hline 
\end{tabular}
\end{center}
\end{table}

\begin{table}
\caption{\label{constraints:scheduling}Scheduling constraints}
\begin{center}
\begin{tabular}{|p{5cm}||c|c|c|c|c|c|}
\hline Constraint & Moz & Cho & Min & Gec & ECL & SiP \\
\hline 

\hline cummulative & \vubec & \ano & \vubec & \ano & \ano & \ano \\
\hline overlaps & \ano & \vubec & \vubec & \vubec & \vubec & \vubec \\
\hline disjunctive & \ano & \ano & \vubec & \vubec & \ano & \ano \\

\hline 
\end{tabular}
\end{center}
\end{table}

\begin{table}
\caption{\label{constraints:extensional}Other constraints}
\begin{center}
\begin{tabular}{|p{5cm}||c|c|c|c|c|c|}
\hline Constraint & Moz & Cho & Min & Gec & ECL & SiP \\
\hline 

\hline extensional 	& \vubec & \ano & \ano & \ano & \trochu & \ano \\
\hline DFA 					& \vubec & \ano & \vubec & \ano & \vubec & \ano \\
\hline hamiltonian circle 	& \vubec & \vubec & \vubec & \ano & \vubec & \ano \\

\hline 
\end{tabular}
\end{center}
\end{table}