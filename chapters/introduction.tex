\chapter{Introduction}
\thispagestyle{myheadings}\markright{$ $Id$ $}

In this thesis we compare several constraint solvers from the perspective of a 
user who is not experienced in the constraint programming. We focus on easiness 
of a learning process of each solver and we measure performance by using benchmarks which 
compare various aspects of examined systems. 

\section{Constraint Satisfaction Problems}
Constraint programming is a programming paradigm which uses constraints to 
describe a solution rather than to program a way of achieving such a solution. 
The constraint can be any relation which can be asserted as true or false -- $X < Y$, 
Billy is older than Johnny, $Z = 5$, etc.  As an example of a problem which can be 
solved using constraint programming we use the Sudoku puzzle. Sudoku is a worldwide 
known logical problem which is easy to explain, its difficulty can be scaled and 
one does not need previous training to solve Sudoku. It makes the problem easier 
to understand for many people and, therefore, it is very popular. Simple rules of 
Sudoku are: There is a given table of size nine times nine. Every field of the table 
contains a number in the range one to nine. In each column all the numbers are different 
(this enforces that every column contains all numbers in range one to nine). In 
each row there are also all numbers different. Finally the same rule which applies 
for columns and rows also restricts three times three sized squares which are in the 
puzzle marked using bolder lines. The Sudoku is prefilled with a couple of values. 
These values help at the beginning of solving and the difficulty can be 
adjusted by their number and placement.

%// vlozit obrazek sudoku
\begin{figure}
\caption{\label{sudoku:example}Example of the Sudoku puzzle}
\begin{sudoku}
|2|5| | |3| |9| |1|.
| |1| | | |4| | | |.
|4| |7| | | |2| |8|.
| | |5|2| | | | | |.
| | | | |9|8|1| | |.
| |4| | | |3| | | |.
| | | |3|6| | |7|2|.
| |7| | | | | | |3|.
|9| |3| | | |6| |4|.
\end{sudoku}
\end{figure}

The way how to describe thr Sudoku puzzle in constraint programming is very 
straightforward. We define the following model:

\begin{enumerate}
\item	There are 81 variables $s_{ij}$, $i,j \in \{1,...,9\}$ which can contain values in range 1 to 9. We 
      arrange them into a two-dimensional array with size 9 $\times$ 9. 
      (The Sudoku is a table sized 9 $\times$ 9 containing values in range $1,...,9$)
\item	For all $i$ in $1,...,9$ the following condition is true: All values of $s_{i \bullet}$ are different 
      (values in each row are different)
\item	For all $j$ in $1,...,9$ the following condition is true: All values of $s_{\bullet j}$ are different 
      (values in each column are different)
\item	For each square the following condition is true: For all $k$, $l$ such that $k$, $l$ 
      are indexes defining the square all values of 
      $s_{kl}$ are different (values in each square are different)
\item	For all prefilled values: $q_{mn} = V$ if and only if the field in column 
      $n$ and row $m$ is prefilled and contains $V$.
\end{enumerate}

These constraints fully describe the Sudoku puzzle problem and as the reader can see 
they do not differ from the commonly known rules. A person solving the Sudoku puzzle can 
use many techniques starting with randomly filling the table and looking if this 
is a good solution (the algorithm using this technique is called GAT -- Generate and Test) 
and ending with generating all possible fillings and correcting the solution if 
something fails (this algorithm is called backtracking). The first approach can miss 
a correct solution. Since the second approach systematically searches all possible 
solutions it has to result in a correct solution; however, it can last enormous 
time to complete it (even on a supercomputer). The secret of a successful solution 
is in the fact that not all numbers can be filled in a specific field. If there is 
prefilled value 8 at position [6,7] it means that in row 6 and in column 7 there
cannot be another number 8. And because of constraint (4) there also cannot be 8 
in the right middle square. A person who does these observations usually writes 
into the destination field all possible values and as an examination of the puzzle 
progresses there are less and less possibilities to fill in. In an easy Sudoku after 
this examination there is at least one field which can be filled with only one number. 
After filling all such fields the solving continues in the same way until 
the entire table is filled. A program which uses constraint programming solves it in the same 
way. For each variable it remembers the range of possible values (we will call it 
a domain). Before the program starts searching for a solution it tries to eliminate as 
many values from the domain as possible. It can reveal that the problem does not have a 
solution (if there is a variable with an empty domain) before a backtracking. It 
it is no surprise that in a user guide to Choco system, one of the available constraint
solvers, it is stated "if you know Sudoku, 
then you know constraint programming." The formal definitions of the constraint satisfaction
problem and the solution of the constraint satisfaction problem follows (the definitions 
are cited from \cite{bartak:ogcp}).

\begin{definition}The {\bf Constraint satisfaction problem} or {\bf CSP} consist of:
\begin{itemize}
	\item a set of variables $X=\{x_1, ..., x_n\}$,
	\item for each variable $x_i$ a finite set $D_i$ of possible values (its {\bf domain}),
	\item and a set of {\bf constraints} restricting the values that the variables can simultaneously take.
\end{itemize} 
\end{definition}

\begin{definition}
A {\bf solution} to a CSP is an assignment of a value from its domain to every variable, in such a way that every constraint is satisfied. We may want to find:
\begin{itemize}
	\item just one solution, with no preference as to which one,
	\item all solutions,
	\item an optimal, or at least a good solution, given some objective function defined in terms of some or all of the variables.
\end{itemize}
\end{definition}

All of the constraint problem can be modeled only by using of the binary constraints (the constraints with two variables). 
The constraint satisfaction problem can be represented as a constraint multi-graph where the variables of the model are the 
nodes of the graph and the constraints over the variables in the model are the edges connecting the apropriate nodes. 
The edge $e(x,y)$ is consistent if for each value of $x$ there exists a value $y$ such
that the constraint is satisfied. As we can see the constraint which is consistent
in one direction does not have to be consistent in the second direction. 

\begin{definition}The constraint satisfaction problem is {\bf arc consistent} if all
of the edges in the constraint graph are consistent in the both directions.
\end{definition}

If the problem cannot be transformed to arc consistent state then it obviously does not
have a solution; however, the arc consistency itself does not ensure that the problem 
can be solved. As an example of such a problem let consider the following: 
We have variables $X,Y,Z$ all with the domain $\{0,1\}$ with the constraints $X \neq Y$, 
 $Y \neq Z$, $Z \neq X$ on them. The problem is arc consistent. If we examine all constraints
 there exists for each possible value of first variable a value of second variable which satisfies 
 the constraint and similary in the second direction. Even though the problem is arc consistent it cannot be solved because there does
 not exists evaluation of all variables which satisfies the all constraints.

\section{Constraint programming}
\label{introduction:constraint-programming}
 
We can use many algorithms to solve the constraint satisfaction problems.
The most common algorithm consists of two phases which are repeated until the
solution is found or we find that there is no solution. The first phase is filtering of
the variable domain by eliminating as many as possible values of the domain. 
We can achieve such filtering by transforming of the problem to the arc consistent state. 
The filtration phase can end in three ways:

\begin{itemize}
 \item The domain of some variable is empty and therefore there is no solution for a given problem.
 \item The domains of all variables have only one element. The algorithm found the solution.
 \item The domains of some variables have more than one element while the problem is arc consistent.
\end{itemize}

If the problem $P$ is arc consistent but we still do not have the solution we perform
the distribution phase. In the distribution phase we introduce a new constraint $c$. We create
two new problems $P \cup \{c\}$ and $P \cup \{\neg c\}$. It is obvious that if there 
exists a solution of the original problem then at least one of the new problems do have 
to have the solution. After this phase we run the filtering phase again.

To construct the constraint $c$ used the distribution phase we usualy pick a variable $x$
and its value $v$. The constraint then can be $x = v$ or $x < v$ (but we can use any constraint). 
By selecting of a proper variable and value we can affect the
time needed to compute the solution. The most common way is to pick a variable with the
smallest domain because we should  be able sooner find out that the variable cannot be evaluated. This strategy
is sometimes called {\em first-fail} strategy. The solvers usualy allows the users to
choose the strategy or implement their own.

The described algorithm is called {\em maintained arc consistency} or MAC. Its
 schematic source code is in figure \ref{labelling-bartak}. The code is cited from 
 \cite{UI:Bartak}. More detailed explanations of algorithms used in CSP solving can be found in \cite{bartak:ogcp}. 

\begin{figure}
\caption{\label{labelling-bartak}The MAC algorithm}
\begin{lstlisting}[language=Pascal]
procedure labelling(V,D,C)
  if all variables from V are assigned then return V
  select not-yet assigned variable x from V
  for each value v from Dx do
    (TestOK, D') := consistent(V,d,C + {x=v})
    if TestOK = true then
      R := labelling(V, D', C)
      if R <> fail then return R
  end for
  return fail
end
\end{lstlisting}
\end{figure}

The optimalization problems can be solved by using of the {\em branch-and-bound} algorithm.
First the algorithm solves the problem $P$ using the standard algorithms. After the problem
is solved the value $f(P)$ of the objective function is computed. Then we derive from the problem 
P a new problem $P'$ with the additional constraint $f(P') < f(P)$ and we run the algorithm on the
new problem. The last found solution which could be satisfied is the expected best solution. 

In reading previous paragraphs, a reader could think that the constraint programming is used 
only as an academic toy for solving Sudoku and for other applications which are useless 
in a real life. In fact the constraint programming is used in various applications. 
A few examples of many contain scheduling, an image recognition, financial modeling, 
planning, vehicle routing, a configuration, computer networks and bioinformatics. 
The constraint programming was also successfully used at NASA in Deep Space 1 experiment. 
Deep Space 1 was a space probe using 12 cutting-edge technologies which were never 
tested in space before. One of these technologies was a remote agent used to plan 
actions of a space vehicle while only general commands were sent to agent. Agent 
used a constraint solver for planning \cite{nasa:ds1-ara}.

\section{Constraint solvers}
A programmer who wants to solve problems using constraint programming can encode the
ideas described in the previous paragraphs or use specialized software, a 
constraint solver. Constraint solver is a system which uses constraint programming 
techniques to solve a given problem. There are many solvers available both commercial 
and freeware. A short list of available systems can be found in appendix 
\ref{list-of-solvers}. A more detailed list is maintained by Roman Bart�k in 
On-line guide to constraints programming at \cite{bartak:los}.

\section{Related work}
In this thesis we compare several constraint solvers. There exist papers and other work
which cover the similar area. 

There exists a paper "A Comparative Study of Eight Constraint Programming Languages 
Over the Boolean and Finite Domains" by A. Fernandez and others \cite{fernandez00} 
which compares several solvers; however, this paper does not focus on the user experience 
with the solvers. They compared solvers by their performance on various benchmarks 
and discussed an implementation of self referential quiz in each solver. We are not 
benchmarking the same sets of solvers and we do not use similar methodology which 
means this thesis conclusions cannot fully replace or update the results from \cite{fernandez00}.

M. Plach� in 2007 wrote the bachelor thesis \cite{placha} focused on the same topic as this thesis.
The thesis compared SICStus Prolog,  ILOG Solver and Gecode/J. She compared the modelling
capabilites of these solvers and showed the ways to debug the models. Finally the speed 
of solvers was measured on several benchmarks. In this thesis we study a wider range of
solvers. We used the same criteria as the mentioned thesis by M. Plach� but additionally
we compare the accessibility of the solvers for user. 

Every year there is held the International Constraint Solver Competition where authors
of solvers can compete. The solvers can be submitted in two categories -- complete 
and incomplete. Complete solvers can prove that the instance of a problem is satisfiable
or not (or find and prove the optimum). As stated in the rules \cite{csp-competition:rules} 
for each solver there is a {\em Boolean capability vector}
which indicates which constraints the solver can handle. Solvers with the same
capability vector can be naturally compared. Solvers with different capabilities 
can be compared on instances which belong to the intersection
of their capabilities, provided it is non-empty. During the competition the solvers
are run in the sandbox environment on a Linux cluster. The task is to find a solution
of as many benchmarks as possible in the smallest time amount.

There exists a library \cite{csplib} of the constraint satisfaction problems which can be used in 
benchmarking and comparing of solver capabilities. The library contains various problems
in several fields -- optimalisation problems, combinatioral problems and so on. We encourage
the reader to try to implement several problems in the chosen solver as a part of learning of
modelling in the solver.

\section{Outline of the thesis}
We described our motivation for this thesis and listed some constraint solvers. 
In the second chapter we define methodology used to examine some of the mentioned 
solvers. The examination consists of two parts -- performance tests and usability 
tests. In the third chapter we will define benchmarks used to performance tests. 
The fourth chapter describes in details each solver, mentions a little from their 
history but mainly focuses on usability and easiness of learning and using the solver. 
In the fifth chapter we discuss the performance tests results and compare the solvers. 
Finally, in the sixth chapter we state a conclusion of the whole examination process.

Ther are four appendixes to this thesis. The content of the included CD is in the
appendix \ref{cdcontents}. The short list of constraint solvers is in the appendix \ref{list-of-solvers}.
The table of supported constraints is in the appendix \ref{constraints}. Finally the
source codes of the benchmarks implemented in the compared solvers are in
the appendix \ref{implementation}.
