\chapter{Introduction}
\thispagestyle{myheadings}\markright{$ $Id$ $}

In this thesis we will compare several constraint solvers using the point of view 
of user who is not experienced in constraint programming. We will focus on easiness 
of learning process of each solver and measure performance using benchmarks which 
test various aspects of examined systems. 

\section{Motivation}
Constraint programming is a programming paradigm which uses constraints to a 
describe solution rather than to program a way of achieving such solution. 
Constraint can be any condition which can be asserted as true or false -- $X < Y$, 
Billy is older than Johnny, $Z = 5$ etc.  As an example of problem which can be 
solved using constraint programming we will use Sudoku puzzle. Sudoku is worldwide 
known logical problem which is easy to explain, its difficulty can be scaled and 
one does not need previous training to solve Sudoku. It makes this problem easy 
to understand for many people and therefore it is very popular. Simple rules of 
Sudoku are: There is given table of size nine times nine. Every field of table 
contains number in range one to nine. In each column all numbers are different 
(this enforces that every column contains all numbers in range one to nine). In 
each row there are also all numbers different. Finally the same rule which applies 
for columns and rows stands for three times three sized squares which are in the 
puzzle marked with bolder lines. The Sudoku is prefilled with a couple of values. 
These values help at the beginning of solving and the difficulty can be 
adjusted by their count and placement.

%// vlozit obrazek sudoku

The way how to describe Sudoku puzzle in the constraint programming is very 
straightforward. We will define following constraints:

\begin{enumerate}
\item	There are 81 variables s which can contain values in range 1 to 9. We will 
      arrange them in a two-dimensional array with size 9 $\times$ 9. 
      (The Sudoku is table sized 9 $\times$ 9 containing values in a range $1..9$)
\item	For all $i$ in 1..9 is true: All values of $s_{i \bullet}$ are different 
      (Values in each row are different)
\item	For all $j$ in 1..9 is true: All values of $s_{\bullet j}$ are different 
      (Values in each column are different)
\item	For each square is true: For all $k$, $l$ such that $k$, $l$ is in square values of 
      $s_{kl}$ are different (Values in each square are different)
\item	For all prefilled values: $q_{mn} = V$ if and only if field with placed in column 
      $n$ and row $m$ is prefilled and contains $V$.
\end{enumerate}

These constraints fully describe Sudoku puzzle problem and as the reader can see 
do not differ from the commonly known rules. A person solving Sudoku puzzle can 
use many techniques starting with randomly filling the table and looking if this 
is a good solution (the algorithm using this technique is called GAT -- Generate and Test) 
and ending with generating all possible fillings and repairing the solution if 
something fails (this algorithm is called backtracking). The first approach can miss 
a correct solution. Since the second approach systematically searches the possible 
solutions it has to result inthe correct solution. However it can last enormous 
time to complete it (even on supercomputer). The secret of successful solution 
is in fact that not all numbers can be filled in specific field. If there is 
prefilled value 8 at position [6,7] it means that in row 6 and in column 7 
cannot be another number 8. And because of constraint (4) there cannot be 8 
also in right middle square. Person who does these observations usually writes 
to the destination field all possible values and as examination of the puzzle 
continues there are less and less possibilities to fill in. In easy Sudoku there 
is after this examination at least one field which can be filled with only one 
number. After filling all such fields solving continues in the same way until 
whole table is filled. Program using constraint programming solves it in the same 
way. For each variable it remembers the range of possible values (we will call it 
domain). Before the program starts searching of solution it tries to eliminate as 
many values from domain as possible. It can reveal that problem does not have a 
solution (if there is some variable with empty domain) before backtracking. It 
it is no surprise that in user guide to Choco system is stated "if you know Sudoku, 
then you know constraint programming."

\section{Constraint programming}
Constraint programming of course consists of more techniques but generally it works 
just like the person solving Sudoku described in the previous paragraph. 

\cite{UI:Bartak}
%% PREPSAT A PRIDAT MAC

Since time complexity 
of searching in possible solutions can be enormous the most important is to eliminate as 
many solution candidates as possible before actual search. Unfortunately eliminating values 
from domain based on given constraints does not guarantee that model can be directly solved. 
This means we have to search the problem space for possible solution. Standard backtracking 
does not work efficiently. It does not have memory so it cannot use results which 
were computed in different branch. It case of fail it also cannot identify problematic 
evaluation of constraint so it has to search whole subtree of given branch to find 
that it cannot result in solution. Constraint programming offers techniques backmarking 
and backjumping which can identify the dead end of current branch and recheck evaluating 
of that variable instead of useless searching through whole subtree. Algorithms 
like looking ahead and forward checking can identify future problems so the search 
algorithm can eliminate bad decisions as soon as possible. Detailed description 
of these techniques would fill whole book and this thesis has another purpose. 
The reader who wants to learn more can found more in \cite{bartak:ogcp}.

From previous paragraphs reader could think that constraint programming is used 
only as an academic toy for solving Sudoku and other applications which are useless 
for real life. In fact constraint programming is used in various applications. 
A few examples of many contains scheduling, image recognition, financial modeling, 
planning, vehicle routing, configuration, computer networks and bioinformatics. 
Constraint programming was also successfully used at NASA in Deep Space 1 experiment. 
Deep Space 1 was a space probe using 12 cutting-edge technologies which were never 
tested in space before. One of these technologies was a remote agent used to plan 
actions of space vehicle while only general commands were sent to agent. Agent 
used for planning constraint solver \cite{nasa:ds1-ara}.

\section{Constraint solvers}
Programmer who wants to solve problem using constraint programming can use in his 
algorithms ideas described in previous paragraphs or use specialized software -- a 
constraint solver. Constraint solver is a system which uses constraint programming 
techniques to solve given problem. There are many solvers available both commercial 
and freeware. As a short list of available systems can be found in table 1.1.

\section{Related work}
There exists a paper "A Comparative Study of Eight Constraint Programming Languages 
Over the Boolean and Finite Domains" by A. Fernandez and others \cite{fernandez00} 
which covers similar area. However this paper does not focus on the user experience 
with the solvers. They compared solvers by their performance on various benchmarks 
and discussed implementation of self referential quiz in each solver. We are not 
benchmarking the same sets of solvers and we do not use similar methodology which 
means this thesis conclusions cannot replace or update results from \cite{fernandez00}.

\section{Outline of the thesis}
We described our motivation for this thesis and listed some constraint solvers. 
In the second chapter we will define methodology used to examine some of mentioned 
solvers. The examination consists of two parts -- performance tests and usability 
tests. In third chapter we will define benchmarks used to performance tests. 
Fourth chapter describes in details each solver, mentions a little from their 
history but mainly focus on usability and easiness of learning and using the solver. 
In fifth chapter we will discuss the performance tests results and compare the solvers. 
Finally in sixth chapter we will state conclusion of whole examination process 
and present decision graph "which solver use in which situation".
