\chapter{Introduction}
\thispagestyle{myheadings}\markright{$ $Id$ $}

In this thesis we compare several constraint solvers from a perspective of a 
user who is not experienced in a constraint programming. We focus on easiness 
of a learning process of each solver and we measure performance by using benchmarks which 
test various aspects of examined systems. 

\section{Motivation}
The constraint programming is a programming paradigm which uses constraints to a 
describe solution rather than to program a way of achieving such solution. 
Constraint can be any condition which can be asserted as true or false -- $X < Y$, 
Billy is older than Johnny, $Z = 5$, etc.  As an example of a problem which can be 
solved using the constraint programming we use Sudoku puzzle. Sudoku is a worldwide 
known logical problem which is easy to explain, its difficulty can be scaled and 
one does not need previous training to solve Sudoku. It makes the problem easier 
to understand for many people and, therefore, it is very popular. Simple rules of 
Sudoku are: There is a given table of size nine times nine. Every field of table 
contains number in range one to nine. In each column all the numbers are different 
(this enforces that every column contains all numbers in range one to nine). In 
each row there are also all numbers different. Finally the same rule which applies 
for columns and rows also restricts three times three sized squares which are in the 
puzzle marked using bolder lines. The Sudoku is prefilled with a couple of values. 
These values help at the beginning of solving and the difficulty can be 
adjusted by their count and placement.

%// vlozit obrazek sudoku

The way how to describe Sudoku puzzle in the constraint programming is very 
straightforward. We define following constraints:

\begin{enumerate}
\item	There are 81 variables which can contain values in range 1 to 9. We 
      arrange them into a two-dimensional array with size 9 $\times$ 9. 
      (The Sudoku is a table sized 9 $\times$ 9 containing values in a range $1..9$)
\item	For all $i$ in 1..9 is true: All values of $s_{i \bullet}$ are different 
      (Values in each row are different)
\item	For all $j$ in 1..9 is true: All values of $s_{\bullet j}$ are different 
      (Values in each column are different)
\item	For each square is true: For all $k$, $l$ such that $k$, $l$ is in square values of 
      $s_{kl}$ are different (Values in each square are different)
\item	For all prefilled values: $q_{mn} = V$ if and only if field with placed in column 
      $n$ and row $m$ is prefilled and contains $V$.
\end{enumerate}

These constraints fully describe the Sudoku puzzle problem and as the reader can see 
do not differ from the commonly known rules. A person solving Sudoku puzzle can 
use many techniques starting with randomly filling the table and looking if this 
is a good solution (the algorithm using this technique is called GAT -- Generate and Test) 
and ending with generating all possible fillings and correcting the solution if 
something fails (this algorithm is called backtracking). The first approach can miss 
a correct solution. Since the second approach systematically searches the possible 
solutions it has to result into the correct solution; however, it can last enormous 
time to complete it (even on supercomputer). The secret of a successful solution 
is in the fact that not all numbers can be filled in a specific field. If there is 
prefilled value 8 at position [6,7] it means that in row 6 and in column 7 
cannot be another number 8. And because of constraint (4) there also cannot be 8 
in the right middle square. A person who does these observations usually writes 
into the destination field all possible values and as an examination of the puzzle 
progresses there are less and less possibilities to fill in. In an easy Sudoku after 
this examination there is at least one field which can be filled with only one number. 
After filling all such fields the solving continues in the same way until 
the entire table is filled. A program which use constraint programming solves it in the same 
way. For each variable it remembers the range of possible values (we will call it 
a domain). Before the program starts searching of solution it tries to eliminate as 
many values from the domain as possible. It can reveal that the problem does not have a 
solution (if there is a variable with an empty domain) before a backtracking. It 
it is no surprise that in a user guide to Choco system it is stated "if you know Sudoku, 
then you know the constraint programming."

\section{Constraint programming}
The constraint programming indeed consists of more techniques but generally it works 
just like the person solving Sudoku described in the previous paragraph.
All constraint satisfaction problems (CSP) can be transformed in such a way that
it contains only binary constraints. Common representation of the CSP is the multigraph 
where nodes are variables and the edges constraints on them. The value of a variable's domain
is {\em supported} if there is not a constraint which collides with such a value.
We can define {\em node consistency} if all values of domain of a node $x$ satisfies
all constraints $c(x,x)$ (unary constraints). The problem is node consistent (NC) if all
nodes in the problem are node consistent. Similary, the edge $e(x,y)$ is arc consistent if
for all values in $D_x$ exists a value of $y$ such that the constraint is satisfied.
This definition does not assure that if the edge $e(x,y)$ is arc consistent then the
edge $e(y,x)$ is arc consistent too. CSP is arc consistent (AC) if all edges are arc consistent in both 
directions. The arc consistency, however, does not guarantee that the problem has
solution. If the problem is not arc consistent then it has no solution for sure.
For example let have variables $X,Y,Z$ with domains $\{0,1\}$ and constraints $X \neq Y$,
 $Y \neq Z$ and $X \neq Z$. The problem is arc consistent but it has no solution.
 
The general way of solving of the CSP is to transform the domains of variables to ensure
the arc consistency of the problem. After the problem is in stable state (last iteration of arc 
consistency algorithm have not changed any domain) we test if the problem is solved.
The problem is solved if each value has one sized domain. If there is an empty domain after 
the arc consistency algorithm we declare the problem as unsatisfiable. This part 
of algorithm is called {\em propagation}. If we have not found the solution in propagation
phase we continue with {\em distribution} phase. In distribution phase we add an additional
constraint $c$ to the problem $P$ and rerun the propagation on two new problems, 
 $P \cup \{ c \}$ and $P \cup \{ \neg c \}$. If there exists a solution it has to be
 contained in at least one of the two new problems. For the usually used constraint
 we choose one variable $x$ and value $v$ from its current domain $D_x$. The constraint 
 $c$ then can be $x = v$ or $x < v$. The choosing algorithm of the proper variable
 $x$ is important. Often we pick the variable with the smallest domain because it leads 
 to possible failure more quickly. Usually the solver offers choices of the algorithm
 to suite all needs. This algorithm is called maintained arc consistency or MAC. Its
 schematic source code is in figure \ref{labelling-bartak}. The code is cited from 
 \cite{UI:Bartak}. More detailed explanations of algorithms used in CSP solving can be found in \cite{bartak:ogcp}.
 
\begin{figure}
\caption{\label{labelling-bartak}The MAC algorithm}
\begin{lstlisting}[language=Pascal]
procedure labelling(V,D,C)
  if all variables from V are assigned then return V
  select not-yet assigned variable x from V
  for each value v from Dx do
    (TestOK, D') := consistent(V,d,C + {x=v})
    if TestOK = true then
      R := labelling(V, D', C)
      if R <> fail then return R
  end for
  return fail
end
\end{lstlisting}
\end{figure}

In reading previous paragraphs, a reader could think that the constraint programming is used 
only as an academic toy for solving Sudoku and for other applications which are useless 
in a real life. In fact the constraint programming is used in various applications. 
A few examples of many contain scheduling, an image recognition, financial modeling, 
planning, vehicle routing, a configuration, computer networks and bioinformatics. 
The constraint programming was also successfully used at NASA in Deep Space 1 experiment. 
Deep Space 1 was a space probe using 12 cutting-edge technologies which were never 
tested in space before. One of these technologies was a remote agent used to plan 
actions of a space vehicle while only general commands were sent to agent. Agent 
used for planning constraint solver \cite{nasa:ds1-ara}.

\section{Constraint solvers}
A programmer who wants to solve problem using constraint programming can use in their 
algorithms ideas described in previous paragraphs or use specialized software, a 
constraint solver. Constraint solver is a system which uses constraint programming 
techniques to solve a given problem. There are many solvers available both commercial 
and freeware. A short list of available systems can be found in table 
\ref{constraints-solvers-list}. More detailed list is maintained by Roman Bart�k in 
On-line guide to constraints programming at \cite{bartak:los}.

\section{Related work}
There exists a paper "A Comparative Study of Eight Constraint Programming Languages 
Over the Boolean and Finite Domains" by A. Fernandez and others \cite{fernandez00} 
which covers similar area; however, this paper does not focus on the user experience 
with the solvers. They compared solvers by their performance on various benchmarks 
and discussed an implementation of self referential quiz in each solver. We are not 
benchmarking the same sets of solvers and we do not use similar methodology which 
means this thesis conclusions cannot replace or update results from \cite{fernandez00}.

Every year there is held the International Constraint Solver Competition where authors
of solvers can compete. The solvers can be submitted in two categories -- complete 
and incomplete. Complete solvers can prove that the instance of problem is satisfiable
or not (or find and prove the optimum). As stated in the rules \cite{csp-competition:rules} 
for each solver there is a {\em Boolean capability vector}
which indicates which constraints the solver can handle. Solvers with the same
capability vector can be naturally compared. Solvers with different capabilities 
can be compared on instances which belong to the intersection
of their capabilities, provided it is non-empty. During the competition the solvers
are run in the sandbox environment on a Linux cluster. The task is to compute solution
of as many benchmarks as possible in the smallest time amount.

There exists a library \cite{csplib} of the constraint satisfaction problems which can be used in 
benchmarking and comparing of solver capabilities. The library contains varying problems
in several fields -- optimalisation problems, combinatioral problems and so on. We encourage
the reader to try to implement several problems in the chosen solver as a part of learning of
modelling in the solver.

\section{Outline of the thesis}
We described our motivation for this thesis and listed some constraint solvers. 
In the second chapter we define methodology used to examine some of mentioned 
solvers. The examination consists of two parts -- performance tests and usability 
tests. In third chapter we will define benchmarks used to performance tests. 
The fourth chapter describes in details each solver, mentions a little from their 
history but mainly focuses on usability and easiness of learning and using the solver. 
In the fifth chapter we discuss the performance tests results and compare the solvers. 
Finally, in the sixth chapter we state a conclusion of the whole examination process 
and present a decision graph "which solver use in which situation".
