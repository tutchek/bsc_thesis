\chapter{Benchmarks}
\verb= $Id$ =

In this chapter we shall define benchmarks which will later be used to examine the properties of each solver. We will be dealing with five different cases which are well known and documented: N-queens problem, magic sequence problem, self referential quiz, quasigroup with holes problem and locating warehouses problem. In this chapter, we will show on these benchmarks the different ways of modelling the problem with solvers.
We will also measure time and the amount of memory needed to compute the solutions to the benchmarking 
problems. Except for the self referential quiz all of the presented problems can be scaled up or down
and can be therefore used to test the robustness of the solver. We shall test the robustness on the 
magic sequence problem; we define the robustness of the solver as the length of
the longest magic sequence which can be computed in the selected solver in a given 
time. The last benchmarking problem -- locating warehouses -- is an optimalisation
problem. The solver not only has to find correct solutions but also has to evaluate the best of the
solutions based on the value of an objective function. For each benchmark 
a general description is presented as is the formal model of the constraint problem and an implementation in
the Essence  programming language. The basics of the language are described in the following 
section.

\section{Essence programming language}
Essence  is a programming language for modelling of combinatorial problems. Is is 
easy enough to understand and so simple that even person who has never seen the
language before can correctly guess what is the expected output of the program. Every program in Essence 
consists of three parts. The first part defines the version of the language, the second
part the used variables and finally, the third part presents the constraints used on the given 
variables. These constrained variables can be integers, booleans and vectors or matrices.
The language supports sums and loops over the variables; however, the bounds of the
sum or loop have to be constant as they are in the process of compiling translated
to a sequence of statements. The program itself can be split into the model definition
and parameters definition parts. Examples in this chapters shows only the model 
definition parts. Files containing the parameter definitions can be found on the included CD.
Tailor tool is the compiler of choice in this case and this program mediates the translation from the Essence language to solver specific language. In the current version it can translate the Essence program to Minion 
input file, FlatZinc and C++ source code using Gecode library. Further description of
the Tailor system is in section \ref{tailor}.

\section{N-queens}
This benchmark is based on a classic chess task. The player has to place eight queens
on the chessboard in a way that none of the queens offends any other. The task
can be scaled to a chessboard of any size. The goal is then place $n$ queens to
onto table of size $n \times n$ in such way that no queen offends any other. A queen
offends all pieces which are placed in the same row, column and diagonal on the chessboard.
The problem is little bit easier if we find out that when we realize that in order to place $n$ queens
on a chessboard with $n$ columns, there has to be one queen per column. Therefore 
we only need to find out in which rows the queens are in each column. We model a
solution of the problem as a vector $q_i$ where $i \in \{1,...,n\}$. To avoid
placing the queens in the same row we simply add constraint that all $q_i$ are 
different. Finally we have to include the diagonals in the solution: two pieces are on the same diagonal, 
if the difference in the horizontal and vertical coordinates is equal. Therefore we 
add the constraint $\left|Q(i) - Q(j)\right| \neq \left|i - j\right|$.
Since both the chessboard and the modes of offence the queen can carry out are symmetric, the solutions are symmetric as well.
We can decrease the number of solutions if we avoid such symmetries.

\subsection{Constraint problem model}
\begin{itemize}
	\item Variables and domains: 
    \begin{itemize}
      \item Positions of queens: $q_1, ..., q_n \in \left\{1, ..., n \right\}$, $q_i$.
    \end{itemize}
	\item Constraints:
    \begin{itemize}
     \item All queens are on different rows: $\forall i,j \in \left\{1, ..., n\right\}: q_i \neq q_j$,
	   \item all queens are on different diagonals: $\forall i,j \in \left\{1, ..., n\right\}: |q_i - q_j| \neq |i - j|$
	   \item optional avoiding of the symmetry: $q_1 < q_n$
	  \end{itemize}
\end{itemize}
The Essence implementation is in the figure \ref{benchmark-essence:nqueens}.

\begin{figure}
\caption{\label{benchmark-essence:nqueens}Implementation of N-Queens Problem in Essence}
\begin{lstlisting}
language ESSENCE' 1.b.a
find queens: matrix indexed by [int(1..n)] of int(1..n)
such that

alldiff(queens),  
forall i: int(1..n). forall j: int(i+1..n).
| queens[i] - queens[j] | != | i - j |
\end{lstlisting} 
\end{figure}

\begin{figure}[ht]
\caption{\label{4queens-solution}Solutions of 4-queens problem}
\begin{center}
\begin{tabular}{cc}
\def\mylist{Qa3,Qb1,Qc4,Qd2}
\setchessboard{setpieces=\mylist,showmover=false,
		pgfstyle=straightmove,
		markmove=c4-a4, %Qc4
		markmove=c4-a2,
		markmove=c4-c1,
		markmove=c4-d3,
		markmove=c4-d4}
\def\mylist{Qa3,Qb1,Qc4,Qd2}
\chessboard[maxfield=d4] & \def\mylist{Qa2,Qb4,Qc1,Qd3}
\setchessboard{setpieces=\mylist,showmover=false}
\def\mylist{Qa2,Qb4,Qc1,Qd3}
\chessboard[maxfield=d4]
\end{tabular}
\end{center}
\end{figure}

\section{Magic sequence}
\label{magic-sequence}
Magic sequence of length $n$ is a sequence of numbers such as $m_i$, $i \in \{0,...,n-1\}$
which satisfies the following condition: The value of $m_k$ is equal to the
number of occurrences of value $k$ in the sequence. For example the sequence $(2\, 1\, 2\, 0\, 0)$ is magic sequence of length five as the condition above is satisfied. The number zero
is twice in the sequence and $m_0$ is equal to two. Similarly one is in the sequence only
once and therefore $m_1$ is one.

\subsection{Constraint problem model}
Model for a magic sequence $m$ of length $k$:

\begin{itemize}
	\item Variables and domains: 
    \begin{itemize}
      \item Magic sequence items: $m_0, ..., m_{k-1} \in  \left\{0, ..., k\right\}$.
    \end{itemize}
	\item Constraints:
	 \begin{itemize}
      \item Value of $m_i$ is $i$ times in the sequence: $\forall i \in 
      {0,...,k-1}: m_i = \sum_{m_j = 1}{1}$.
    \end{itemize}
\end{itemize}
  
In case that solver does not support the constraint $m_i = \sum_{m_j = 1}{1}$, we 
can use an alternative model: 

\begin{itemize}
	\item Values and domains:
  \begin{itemize} 
	 \item Magic sequence items: $m_0, ..., m_{k-1} \in \left\{0, ..., k\right\}$,
	 \item auxiliary variables: $\forall i,j \in {0, ..., k-1}: \mathrm{aux}_{ij}$.
	\end{itemize}
	\item Constraints:
	 \begin{itemize}
    \item $\mathrm{aux}_{ij} = 1$ if and only if $m_j = 0$: $\forall i,j \in \left\{0, ..., k-1\right\}: 
          (\mathrm{aux}_{ij} = 1) \Leftrightarrow (m_j = i)$,
	  \item the value of the items of the magic sequence corresponds to the sum of some auxiliary variables: 
          $\forall i \in \left\{0, ..., k-1\right\}: m_i = \sum_{j=0}^{k-1}{\mathrm{aux}_{ij}}$.
    \end{itemize}
\end{itemize}
The Essence implementation is in figure \ref{benchmark-essence:mseq}.

\begin{figure}
\caption{\label{benchmark-essence:mseq}Implementation of Magic Sequence Problem in Essence}
\begin{lstlisting}
language ESSENCE' 1.b.a
find s : matrix indexed by [int(0..n-1)] of int(0..n)
such that
  forall i : int(0..n-1).
        ( s[i] = (sum j : int(0..n-1). (s[j] = i)))
\end{lstlisting} 
\end{figure}

\section{Self-referential quiz}

Self-referential quiz is a quiz where the answers to the questions depend on the answers
to other questions in the same quiz. There is only one valid answer for each question.
Typical question in such quiz can be: 

\begin{enumerate}
  \item First question where the answer is $A$: \\
    (A) 1 (B) 2 (C) 3 (D) 4 (E) there is no question with answer A
  \item Answer to this question: \\ 
    (A) A (B) B (C) C (D) D (E) E
\end{enumerate}

These quizzes are best modelled using reified constraints. Reified constraint is a constraint
in the form $(C \Leftrightarrow x) \& x \in \left\{0,1\right\}$. The ways how to construct
such quizzes are described in the article by Maja Bubalo \cite{jios:bubalo}.
The quiz assignment follows:

\begin{enumerate}
	\item The first question to which the answer is A:\\
		(A) 4 (B) 3 (C) 2 (D) 1 (E) none of above
	\item The only two consecutive questions with identical answers: \\
		(A) 3 and 4 (B) 4 and 5 (C) 5 and 6 (D) 6 and 7 (E) 7 and 8
	\item The next question with answer A: \\
		(A) 4 (B) 5 (C) 6 (D) 7 (E) 8
	\item The first even numbered question with answer B: \\
		(A) 2 (B) 4 (C) 6 (D) 8 (E) 10 
	\item The only odd numbered question with answer C: \\
		(A) 1 (B) 3 (C) 5 (D) 7 (E) 9
	\item A question with answer D: \\
		(A) comes before this one, but not after this one (B) comes after this one, but not before this one (C) comes before and after this one (D) does not occur at all (E) none of the above
	\item The last question to which the answer is E: \\
		(A) 5 (B) 6 (C) 7 (D) 8 (E) 9
	\item The number of questions to which the answer is a consonant: \\
		(A) 7 (B) 6 (C) 5 (D) 4 (E) 3
	\item The number of questions to which the answer is a vowel: \\
		(A) 0 (B) 1 (C) 2 (D) 3 (E) 4
	\item The answer to this question is: \\
		(A) A (B) B (C) C (D) D (E) E
\end{enumerate}

We model the quiz as a table of boolean variables with five columns (A, B, C, D, E) 
and ten rows, one for each question. The value in the column $i$ and row $j$ is
{\em true} if and only if the answer to the question $j$ is $i$. Because there is only one
answer possible to each question, we constraint the rows of the table to contain
only one {\em true}. The test has three solutions. One of them is showed in the figure 
\ref{srq-solution-table}, the other two solutions differ in the answer to the question
10 which can be also B or C.  
 
\begin{table}[ht]
\caption{\label{srq-solution-table}Solution of a Self Referential Quiz}
\begin{center}
\begin{tabular}{|c|c|c|c|c|c|}
\hline Question & A & B & C & D & E \\
\hline\hline 1 & 0 & 0 & 1 & 0 & 0 \\
\hline 2 & 1 & 0 & 0 & 0 & 0 \\
\hline 3 & 0 & 1 & 0 & 0 & 0 \\
\hline 4 & 0 & 1 & 0 & 0 & 0 \\
\hline 5 & 1 & 0 & 0 & 0 & 0 \\
\hline 6 & 0 & 1 & 0 & 0 & 0 \\
\hline 7 & 0 & 0 & 0 & 0 & 1 \\
\hline 8 & 0 & 1 & 0 & 0 & 0 \\
\hline 9 & 0 & 0 & 0 & 0 & 1 \\
\hline 10 & 0 & 0 & 0 & 1 & 0 \\
\hline 
\end{tabular}
\end{center}
\end{table}

\subsection{Constraint problem model}
	\begin{itemize}
  \item Variables and domains: 
    \begin{itemize}
      \item Answer to the questions: $s_{1|1}, s_{1|2}, ..., s_{10|4}, s_{10|5} \in \{0, 1\}$.
    \end{itemize}
	\item Constraints:
	 \begin{itemize}
    \item There is exactly one value 1 in the row: \\
      $\forall i \in \{1, \ldots, 10\}: \left(\sum_{j \in \{1, ..., 5\}}{s_{i|j}}\right) = 1$,
    \item question 1, A to D: \\
      $\forall i \in \{1,...,4\}: (s_{1|i} = 1) \Leftrightarrow (s_{4-i+1|1} = 1 \mand (\forall j \in \{1,...,4-i\}: s_{j|1} = 0))$,
    \item question 1, E: \\
      $(s_{1|5} = 1) \Leftrightarrow (\forall j \in \{1,...,4\}: s_{j|1} = 0)$,
    \item question 2: \\
      $\forall i \in \{1,...,5\}: (s_{2|i} = 1) \Leftrightarrow (\forall j \in \{1,...,5\}: s_{(3+i-1)|j} = s_{3+i|j})$,
    \item question 3: \\
      $\forall i \in \{1,...,5\}: (s_{3|i} = 1) \Leftrightarrow (s_{4+i-1|1} = 1 \mand (\forall j \in \{4..2+i\}: s_{j|1} = 0 ) )$,
    \item question 4: \\
      $\forall i \in \{1,...,5\}: (s_{4|i} = 1) \Leftrightarrow (s_{2i|2} = 1 /\ ( \forall j \in \{1..i-1\}: s_{2j|2} = 0 )$,
    \item question 5: \\
      $\forall i \in \{1,...,5\}: (s_{5|i} = 1) \Leftrightarrow (s_{2i-1|3}=1)$
    \item question 6, A: \\
      $\forall i \in \{1,...,5\}: (s_{6|1} = 1) \Leftrightarrow (\exists j \in \{1,...,5\}: s_{j|4} = 1 \mand \forall j \in {7,...,10}: s_{j|4} = 0)$
    \item question 6, B: \\
      $\forall i \in \{1,...,5\}: (s_{6|2} = 1) \Leftrightarrow (\exists j \in \{7,...,10\}: s_{j|4} = 1 \mand \forall j \in {1,...,5}: s_{j|4} = 0)$
    \item question 6, C: \\
      $\forall i \in \{1,...,5\}: (s_{6|3} = 1) \Leftrightarrow (\exists j \in \{1,...,5,7,...,10\}: s_{j|4} = 1)$
    \item question 6, D: \\
      $\forall i \in \{1,...,5\}: (s_{6|4} = 1) \Leftrightarrow (\forall j \in \{1,...,10\}: s_{j|4} = 0)$
    \item question 6, E: \\
      $\forall i \in \{1,...,5\}: (s_{6|5} = 1) \Leftrightarrow (s_{6|4} = 1)$
    \item question 7: \\
      $\forall i \in \{1,...,5\}: (s_{7|i} = 1) \Leftrightarrow (s_{i+4|5} = 1) \mand ( \forall j \in \{i+4+1...10\}: s_{j,5} = 0 ) )$
    \item question 8: \\
      $\forall i \in \{1,...,5\}: (s_{8|i} = 1) \Leftrightarrow \left(\sum_{j=1}^{10}{\left(s_{j|2} + s_{j|3} + s_{j|4}\right)} = 7-i+1 \right)$
    \item question 9: \\
      $\forall i \in \{1,...,5\}: (s_{0|i} = 1) \Leftrightarrow \left(\sum_{j=1}^{10}{\left(s_{j|1} + s_{j|5}\right)} = i-1 \right)$. 
   \end{itemize}
	\end{itemize}
The Essence implementation is in the figure \ref{benchmark-essence:srq}.

\begin{figure}
\caption{\label{benchmark-essence:srq}Implementation of the Self Referential Quiz in Essence}
\begin{lstlisting}
language ESSENCE' 1.b.a
find s : matrix indexed by [int(1..10), int(1..5)] of bool
such that
$ there is only one answer to each question and there is not any unanswered question
 forall row : int(1..10). ((sum col : int(1..5). s[row,col]) = 1),
$ Question 1
$ A to D
  forall col : int(1..4). ( (s[1,col] = 1) <=> ( (s[(4-col+1),1] = 1) /\ ( forall row : int(1..(4-col)). (s[row,1] = 0) ) ) ),
$ E
  (s[1,5] = 1) <=> (forall row : int(1..4). (s[row,1] = 0)),

$ Question 2
  forall col : int(1..5). ( (s[2,col] = 1) <=> ( forall col2: int(1..5). (s[3+col-1,col2] = s[3+col,col2]) ) ),
$ Question 3
  forall col : int(1..5). ( (s[3,col] = 1) <=> ( (s[(4+col-1),1] = 1) /\ ( forall row : int (4..2+col). s[row,1] = 0 ) ) ),
$ Question 4
  forall col : int(1..5). ( (s[4,col] = 1) <=> ( (s[col*2,2] = 1) /\ ( forall row : int(1..(col-1)). s[row*2,2] = 0 ) ) ),
$ Question 5
  forall col : int(1..5). ( (s[5,col] = 1) <=> (s[2*col-1,3]=1) ),
$ Question 6
  (s[6,1] = 1) <=> ( ( exists row : int(1..5). s[row,4] = 1 ) /\ ( forall row : int (7..10). s[row,4] = 0 ) ),
  (s[6,2] = 1) <=> ( ( exists row : int(7..10). s[row,4] = 1 ) /\ ( forall row : int (1..5). s[row,4] = 0 ) ),
  (s[6,3] = 1) <=> ( ( exists row : int(7..10). s[row,4] = 1 ) /\ ( exists row : int (1..5). s[row,4] = 1 ) ),
  (s[6,4] = 1) <=> ( forall row : int (1..10). s[row,4] = 0 ),
  (s[6,5] = 1) <=> (s[6,4] = 1),
$ Question 7
  forall col : int(1..5). ( (s[7,col] = 1) <=> ( (s[col+4,5] = 1) /\ ( forall row : int (col+4+1..10). s[row,5] = 0 ) ) ),
$ Question 8
  forall col: int(1..5). ( (s[8,col] = 1) <=> ( ( sum row: int(1..10). (s[row,2] + s[row,3] + s[row,4]) ) = (7-col+1) ) ),
$ Question 9
  forall col: int(1..5). ( (s[9,col] = 1) <=> ( ( sum row: int(1..10). (s[row,1] + s[row,5]) ) = (col-1) ) )
$ Constraints for question 10 are useless
\end{lstlisting} 
\end{figure}

\section{Quasigroup with holes}
Quasigroup or latin square is a table of size $n \times n$ filled with numbers in the
range $1..n$ such as that all values in each row and in each column are unique. There
can be also additional condition on the items of quasigroup, for example that the 
items on the main diagonal have to be even. The task is to fully fill the 
given partly filled quasigroup. The completed quasigroup have to satisfy all previously
stated constraints. This problem is called {\em quasigroup completion problem} or QCP.
Unfortunately, this benchmark does not provide a consistent result. Some partial fillings
can be solved surprisingly easily while other can be extremely demanding; 
some can even be impossible to solve. The crucial problem is that the determining whether the problem can
be solved or not is an NP-hard task. Therefore we cannot determine with certainty whether the problem is only
too hard for the solver or the solution does not exist at all and the solver has to search 
through enormous space of the particular state. To avoid this uncertainty, we use a modification of
QCP called {\em quasigroup with holes} or QWH. First we generate fully filled quasigroup
which satisfies the given conditions and then we exclude some of the values; this quasigroup
with holes is the new assignment for QCP. We have a guarantee that the assignment is correct
and that it can be solved. Generation of QWH assignments was studied by D. Achlioptas
et al. who found out that the difficulty of finding a solution of such problem depends on the size of the
so called backbone \cite{Achlioptas00generatingsatisfiable}. The backbone is a set 
of positions in the quasigroup which have the same value in all solutions.
If the size of the backbone is close to $0 \%$, there are many different solutions and
the solver can find some "by accident". On the other, hand if the backbone is close
to $100 \%$, there is only one solution and all constraints lead towards it.
We can therefore expect interesting behaviour for the backbone of the size $50 \%$,
The experiments \cite{Achlioptas00generatingsatisfiable} showed that this interesting 
behaviour is somewhere near the $30 \%$.

The quasigroups we use have no additional conditions. The used quasigroups are produced
by generators {\em lsencode} created by Carla Gomez and {\em walksat} Henry Kautz. 
 
\subsection{Constraint problem model}
Model quasigroup of order $n$. Assignment values are in the vector $\mathrm{data}_{ij}$:

\begin{itemize}
	\item Variables and domains: 
    \begin{itemize}
      \item Quasigroup items: $q_{11}, ..., q_{nn} \in \left\{1, ..., n\right\}$.
    \end{itemize}
	\item Podm�nky:
	 \begin{itemize}
    	\item All items in one row are unique: $\forall i \in \left\{1, ..., n\right\}: \forall j,k \in \left\{1, ..., n\right\}: q_{ij} \neq q_{ik}$, 
    	\item all items in one column are unique: $\forall i \in {1, ..., n}: \forall j,k \in {1, ..., n}: q_{ji} \neq q_{ki}$ 
    	\item some items of the quasigroup are preassigned: $\mathrm{data}_{ij} \textnormal{ definov�no } \Leftrightarrow (q_{ij} = \mathrm{data}_{ij})$
	 \end{itemize}
\end{itemize}
The Essence implementation is in the figure \ref{benchmark-essence:qwh}.

\begin{figure}
\caption{\label{benchmark-essence:qwh}Implementation of Quasigroup With Holes Problem in Essence}
\begin{lstlisting}
language ESSENCE' 1.b.a
letting nDomain be domain int(1..n)
find qcp : matrix indexed by [nDomain,nDomain] of nDomain
such that
  forall i : nDomain. alldiff(qcp[i, nDomain]),
  forall i : nDomain. alldiff(qcp[nDomain,i])
\end{lstlisting} 
\end{figure}

\section{Locating warehouses}
Lets assume that we want to help a hypothetical business company with the decision which
warehouses should be built for their shops and which warehouses should supply which
shop. The main criterion is in this case the cost of the solution. The cost has two components. The first one
is payment of the constant cost for opening a new store. The second component is the price for the distribution of
goods from a warehouse to the shop. The price varies and is different for all pairs (warehouse, shop). 
Each possible warehouse has a defined maximum capacity which accounts for the number
of shops which can be supplied from this warehouse. As the last condition of the task, we state that all shops have to be supplied. Our task is to choose the solution with minimal total cost.

Both maximum capacities of warehouses and a table with
prices for supplying each of the pairs (warehouse, store) are the required input parameters for the solver which computes the vector
 $s_i$, $i \in \{1,...,\#\mathrm{ of stores}\}$ where $s_i$ indicates which warehouse
 supplies the shop $i$.

\subsection{constraint problem model}
We can build $W$ warehouses. The price for opening of a new warehouse is fixed and 
stored in the parameter $\mathrm{openCost}$. We also have $S$ shops. The maximum capacity
of warehouses is given by vector $w$, where $w_i$ is the maximum number of shops which
can be supplied by the warehouse $i$, $i \in \{1,...,W\}$.
Finally, we have a matrix of supply costs $\mathrm{supplyCost}$ where $\mathrm{supplyCost}_{ij}$ 
translates as the cost of supplying the shop $j$ from warehouse $i$.

	\begin{itemize}
  \item Variables and domains:
    \begin{itemize}
      \item Total cost -- value of objective function: $\mathrm{totalCost} \in \mathbb{N} $,
      \item number of opened shops: $\mathrm{numberOpen} \in \{0, W\}$,
      \item indication whether the shop is open: $\mathrm{open}_1, ..., \mathrm{open}_{W} \in \{0, 1\}$,
      \item indication which warehouse supplies which shop: $\mathrm{supplier}_1, ..., \mathrm{supplier}_{S} \in \{1, ..., W\}$,
      \item indication the supply cost for given shop: $\mathrm{cost}_1, ..., \mathrm{cost}_{S} \in \mathbb{N}$,
      \item total supply cost: $\mathrm{costSum} \in \mathbb{N}$.
    \end{itemize} 
	\item Constraints:
  	\begin{itemize}
  	 \item Objective function: $\mathrm{totalCost} = \mathrm{costSum} + \mathrm{numberOpen} \cdot \mathrm{openCost}$,
  	 \item total supply cost: $\mathrm{costSum} = \sum_i{\mathrm{cost}_i}$,
  	 \item number of opened warehouses: $\mathrm{numberOpen} = \sum_i{\mathrm{open}_i}$,
  	 \item maximal capacity of each warehouse: $\forall i \in \{1,...,W\}: w_i \geq \sum_{\mathrm{supplier}_j = i}{1}$,
  	 \item the warehouse is open if it suppplies at least one shop: $\forall i \in \{1,...,W\}: \left(\mathrm{open}_i = 1\right) \Leftrightarrow \left(\left(\sum_{\mathrm{supplier}_j = i}{1}\right) > 0\right)$,
  	 \item supply cost computation: $\forall i \in {1,...,S}, \forall j \in \{1,...,W\}: \left(\mathrm{supplier}_i = j\right) \Rightarrow \left(\mathrm{cost}_i = \mathrm{supplyCost}_{ij} \right)$.
     
     %$ \sum_{w \textnormal{ is open}}\left(c+\sum_{s \textnormal{ is suplied by } w} \mathrm{SC}(s,w) \right)$
    \end{itemize}
  \end{itemize} 
The Essence implementation is in figure \ref{benchmark-essence:lw}.

\begin{figure}
\caption{\label{benchmark-essence:lw}Implementation of Locating Warehouses Problem in Essence}
\begin{lstlisting}
language ESSENCE' 1.b.a

given Capacity : matrix indexed by [WarehousesRANGE] of int(0..numberOfStores)
given StoreWarehouseCost : matrix indexed by [StoresRANGE,WarehousesRANGE] of CostRANGE
letting CostRANGE be domain int(0..maxCost)
letting StoresRANGE be domain int(0..numberOfStores-1)
letting WarehousesRANGE be domain int(0..numberOfWarehouses-1)

find 
  TotalCost : CostRANGE,
  Open : matrix indexed by [WarehousesRANGE] of int(0..1),
  NumberOpen : int(0..numberOfWarehouses),
  Supplier : matrix indexed by [StoresRANGE] of WarehousesRANGE,
  Cost : matrix indexed by [StoresRANGE] of CostRANGE,
  SumCost : CostRANGE
  
minimising TotalCost

such that
  TotalCost = SumCost + NumberOpen * warehouseCost,
  SumCost = sum j : StoresRANGE. (Cost[j]),
  NumberOpen = sum j : WarehousesRANGE. (Open[j]),

  forall i : WarehousesRANGE.  
	(Capacity[i] >= (sum j : StoresRANGE. (Supplier[j] = i))),

  forall i : WarehousesRANGE.  
	(((sum j : StoresRANGE. (Supplier[j] = i)) > 0) => (Open[i] = 1) ),

  forall i : WarehousesRANGE.  
	(((sum j : StoresRANGE. (Supplier[j] = i)) = 0) => (Open[i] = 0) ),
  
  forall i : StoresRANGE. forall j : WarehousesRANGE. ( (Supplier[i] = j) => (Cost[i] = StoreWarehouseCost[i,j]) )
\end{lstlisting}
\end{figure}