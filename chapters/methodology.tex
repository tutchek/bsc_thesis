\chapter{Methodology}
\thispagestyle{myheadings}\markright{$ $Id$ $}

In the introduction we explained what constraint solvers are and presented several 
examples of them. In the rest of the thesis we focus on six of them -- Mozart/Oz, 
Choco, Minion, Gecode, \eclipse and SICStus Prolog. The first solver is a 
professional commercial solution and the others are freely available open source 
products. The purpose of this thesis is to help new users with choosing of the right solver. 
Therefore we study complexity of learning and using of each solver and their 
performance and abilities. We test solvers which use various programming languages 
and paradigms. The imperative paradigm is represented by a C++ library Gecode and a Java 
library Choco. Users experienced in a logical programming might find interesting 
SICStus Prolog or \eclipse. Mozart is implementation of Oz, a multi-paradigm 
programming language. With these solvers the users can use their current experience and 
just learn an API of the constraint library. The other solvers are configured by 
a solver-specific problem description language. This fact is both advantage and disadvantage. 
As for disadvantage, we must accept that user cannot use their experience with 
existing programming languages and has to learn new concepts; however, a specialized 
language for describing constraint problems can be more accessible for users who 
do not have any programming experiences but they need to solve the given problem.
A general overview of the examination follows.

First, we examine all solvers from a perspective of a user experienced in the given 
programming language but inexperienced in using of the solver. In case of OPL 
and Minion we expect that user has general computers knowledge and is able 
to describe a given problem in constraints. The first examination tries to answer 
a question how difficult it is to learn to use the solver. We model problems 
described in the third chapter and look for constraints which cannot be modelled 
and we describe possible solutions. 

Secondly, a quality of documentation is also important criterion. A solver can be 
the best of all, but it is useless if the user cannot understand the usage. 
The quality of documentation is perceived subjectively and cannot be measured 
exactly. This means that any evaluation is only informational, although it should be 
considered. As a documentation we accept a user guide as well as all other available 
guides, documents, web pages or a doxygen style documentation. An existence of user 
forums or mailing lists is also an important part of learning of new systems. We 
mention a couple of existing solvers. 

Last but not least, we aim for debugging. There are two areas which can be 
debugged - correctness of program and correctness of model. The correctness of 
program means that the program does what it should do, that it handles all inputs as a programmer expects and so on. 
The correctness of model stands for an accurate describtion of given modely. User should be able to 
inspect variables, visualise a decision tree of search and more other information. 
We discuss the ways how a solver informs about mistakes (and how much descriptive the 
information is), the tools provided with solver to debug the program and similarly the 
tools which can be used to debug model correctness.

When user masters the solver and uses it to solve real problems, the time and space 
efficiency of used algorithms matters. We neither examine source codes, nor analyze 
time complexity of used algorithms. Instead we measure the time needed to 
load and the time to solve the problem. If a solver cannot provide such information 
we measure only a total time. We also measure an amount of consumed memory 
during the program execution. All measurements is performed several times 
and averaged to avoid randomness. A robustness test is also performed. 
In the robustness test we set limit one hour and try to compute the longest magic 
sequence (reader can find definition of a magic sequence in the section \ref{magic-sequence}) 
in given time. We use the models created in the process described in the previous 
paragraph. In \cite{fernandez00} authors have sent models to the solvers' authors and 
have given them a chance to modify them to achieve the best performance of their solvers. 
We focus on first-time users of a solver, so we use our own models which are 
not perfect and, more imporantly, not tuned for any particular solver. All solvers 
are tested on Debian 4.0 Linux with kernel 2.6.18 on Pentium 4, 3GHz (single core 
with hyper threading).