\chapter{Benchmarks}
\thispagestyle{myheadings}\markright{$ $Id$ $}

\section{Essence}
Essence is constraint language for specifying combinatorial problems. We will use it as common problem implementation tool. Essence language can be directly translated to Gecode program or Minion input file using Taylor tool. Essence is general language which is easy to write and easy to learn and understand. Essence program consists of three parts. First part is definition of the version of the language, second part defines used variables and third part defines the constraints. Supported data types of variables are Booleans, integers and matrices. For example if we want to declare Boolean variable firstVar and integer variable secondVar whose values is within range from 5 to 9 the corresponding Essence statement would be

\lstloadlanguages{C++}

\definecolor{bgcolor}{rgb}{1,1,0.6} % definice barvy

\lstset{         % nastaveni prostredi listings
  language=C++,
	basicstyle=\tiny,
  keywordstyle=\bfseries,
  identifierstyle=\itshape,
  stringstyle=\ttfamily,
  numbers=none,
  numberstyle=\tiny,
  numbersep=5pt,
  frameround=ffff,
  extendedchars=true,
  tab=~,
  tabsize=2,
  frame=single,
  captionpos=b,
  showspaces=false,
  showstringspaces=false,
  breaklines=true
}



\begin{lstlisting}
find firstVar
\end{lstlisting} 



\section{N-queens}
N-queens problem is a classical constraint satisfaction task. The task is to place n queens on an $n \times n$ chessboard such that none of them is able to capture any other using standard chess queen's moves. Since there are n columns on the chessboard there have to be one queen in every column. This results in the simplification of the task. We do not have to specify problem as two-dimensional matrix. We can specify it as an n-dimensional vector Q where nth position of the vector represents the row of nth queen. The constraint "one queen does not attack any other" can be fulfilled by two constraints. The queen attacks all fields in row where it stands. Therefore all queens have to be on different rows and we can specify all values of Q are different. The queen also attacks all fields on diagonals which contains the field where the queen stands. Field $[X ; Y]$ is on the same diagonal with the field $[A ; B]$ if and only if the distance $AX$ is the same as the distance of $BY$ (see figure \ref{4queens-solution}). This gives us second constraint for queen $Q_1$ placed on field $[Q_{1x};Q_{1y}]$ and queen $Q_2$ placed on field $[Q_{2x};Q_{2y}]$ must be true following formula: $|Q_{1x}-Q_{2x}| \neq |Q_{1y} - Q_{2y}|$. There exists symmetric solutions. To avoid most of them we can optionally specify constraint $Q_1 < Q_n$.

\subsection{Formal definition}
Let have a vector Q of the length n with values in domain 1...n which satisfies following constraints:

\begin{enumerate}
\item $\forall i,j=1,..,n: i \neq j \Leftrightarrow Q(i) \neq Q(j)$ [all values of Q are different]
\item $\forall i,j=1,..,n: i \neq j \Leftrightarrow  |i-j| \neq |Q(i) - Q(j)|$  [two queens are not on the same diagonal]
\item $Q(1) < Q(n)$  [optional constraint to avoid symmetry]
\end{enumerate}

Any vector Q which satisfies defined constraints is solution of the n-queens problem.

\subsection{Example solutions}
For n = 4 we have 4-queens problem. The solutions of 4-queens problem are $[2,4,1,3]$, $[3,1,4,2]$. The solution are visualised on the chessboard on the figure \ref{4queens-solution}. For one queen there are shown the fields which are attacked. There are two solutions found but as we can see the solutions are symmetrical.

\begin{figure}[ht]
\caption{\label{4queens-solution}Solutions of 4-queens problem}
\begin{center}
\begin{tabular}{cc}
\def\mylist{Qa3,Qb1,Qc4,Qd2}
\setchessboard{setpieces=\mylist,showmover=false,
		pgfstyle=straightmove,
		markmove=c4-a4, %Qc4
		markmove=c4-a2,
		markmove=c4-c1,
		markmove=c4-d3,
		markmove=c4-d4}
\def\mylist{Qa3,Qb1,Qc4,Qd2}
\chessboard[maxfield=d4] & \def\mylist{Qa2,Qb4,Qc1,Qd3}
\setchessboard{setpieces=\mylist,showmover=false}
\def\mylist{Qa2,Qb4,Qc1,Qd3}
\chessboard[maxfield=d4]
\end{tabular}
\end{center}
\end{figure}


\subsection{Implementation in Essence}
The implementation of Magic sequence problem can be found in figure \ref{nqueens-essence-implementation}.

\lstloadlanguages{C++}

\definecolor{bgcolor}{rgb}{1,1,0.6} % definice barvy

\lstset{         % nastaveni prostredi listings
  language=C++,
	basicstyle=\tiny,
  keywordstyle=\bfseries,
  identifierstyle=\itshape,
  stringstyle=\ttfamily,
  numbers=none,
  numberstyle=\tiny,
  numbersep=5pt,
  frameround=ffff,
  extendedchars=true,
  tab=~,
  tabsize=2,
  frame=single,
  captionpos=b,
  showspaces=false,
  showstringspaces=false,
  breaklines=true
}


\begin{figure}[ht]
\caption{\label{nqueens-essence-implementation}Implementation of Magic sequence in Essence}
\begin{lstlisting}
language ESSENCE' 1.b.a
find queens: matrix indexed by [int(1..n)] of int(1..n)
such that
forall i: int(1..n). forall j: int(i+1..n).
  | queens[i] - queens[j] | != | i - j |,
  alldiff(queens)
\end{lstlisting} 
\end{figure}

\section{Magic sequence}
Magic sequence $MS$ is special sequence of the length $n$ which satisfies the constraint that number $x$ is in $MS$ contained exactly $MS(x)$ times (vector values are numbered from 0). It is easy to estimate that the vector values have to be in domain $0,\dots,n-1$. The value cannot be greater than $n$, because in such case it would mean that some number is contained more than $n$ times in sequence of the length $n$. It cannot be greater than $n-1$ either. If there would be some $x$ such that $MS(x) = n$, there must be $MS(n) \geq 1$. But since the vector is indexed starting from 0 it means it has maximal index $n-1$ so index $n$ is out of range.

\subsection{Formal definition}
Let have a vector $MS$ of length $n$ which satisfies following constraint:

\begin{itemize}
\item $MS(i)= \sum_{MS(j)=i} 1$
\end{itemize}

\subsection{Implementation in Essence}
The implementation of Magic sequence problem can be found in figure \ref{ms-essence-implementation}.

\lstloadlanguages{C++}

\definecolor{bgcolor}{rgb}{1,1,0.6} % definice barvy

\lstset{         % nastaveni prostredi listings
  language=C++,
	basicstyle=\tiny,
  keywordstyle=\bfseries,
  identifierstyle=\itshape,
  stringstyle=\ttfamily,
  numbers=none,
  numberstyle=\tiny,
  numbersep=5pt,
  frameround=ffff,
  extendedchars=true,
  tab=~,
  tabsize=2,
  frame=single,
  captionpos=b,
  showspaces=false,
  showstringspaces=false,
  breaklines=true
}


\begin{figure}[ht]
\caption{\label{ms-essence-implementation}Implementation of Magic sequence in Essence}
\begin{lstlisting}
language ESSENCE' 1.b.a
find s : matrix indexed by [int(0..n-1)] of int(0..n)
such that
  forall i : int(0..n-1).
        ( s[i] = (sum j : int(0..n-1). (s[j] = i)))
\end{lstlisting} 
\end{figure}
        
\section{Self Referential Quiz}

Self referential quiz is a set of questions whose answers depends one on another. Typical questions in this type of quizzes are "1. First question with answer A is: A) 1, B) 2, C) 3, D) 4, E) 5; 2. Answer to this question is: A) A, B) B, C) C, D) D, E) E.". Reader can find methods to create such quiz in \cite{jios:bubalo}. We will use the same quiz as Fernandez which contains following ten questions:

\begin{enumerate}
	\item The first question whose answer is A is:\\
		(A) 4 (B) 3 (C) 2 (D) 1 (E) none of above
	\item The only two consecutive questions with identical answers are: \\
		(A) 3 and 4 (B) 4 and 5 (C) 5 and 6 (D) 6 and 7 (E) 7 and 8
	\item The next question with answer A is: \\
		(A) 4 (B) 5 (C) 6 (D) 7 (E) 8
	\item The first even numbered question with answer B is: \\
		(A) 2 (B) 4 (C) 6 (D) 8 (E) 10 
	\item The only odd numbered question with answer C is: \\
		(A) 1 (B) 3 (C) 5 (D) 7 (E) 9
	\item A question with answer D: \\
		(A) comes before this one, but not after this one (B) comes after this one, but not before this one (C) comes before and after this one (D) does not occur at all (E) none of the above
	\item The last question whose answer is E is: \\
		(A) 5 (B) 6 (C) 7 (D) 8 (E) 9
	\item The number of questions whose answers are consonants is: \\
		(A) 7 (B) 6 (C) 5 (D) 4 (E) 3
	\item The number of questions whose answers are vowels is: \\
		(A) 0 (B) 1 (C) 2 (D) 3 (E) 4
	\item The answer to this question is: \\
		(A) A (B) B (C) C (D) D (E) E
\end{enumerate}
This quiz can be implemented as table 10 times 5 Boolean variables. Value in the row x and column y is true if and only if answer to question x is y (A=1, B=2, C=3, D=4, E=5). Because there is answer to every question and also every question has only one answer we have to set the constraint that there is only one true value in a row. 

\subsection{Solution}
There exists only one solution which is shown in table \ref{srq-solution-table}.

\begin{table}[ht]
\caption{\label{srq-solution-table}Solution of Self Referential Quiz}
\begin{center}
\begin{tabular}{|c|c|c|c|c|c|}
\hline Question & A & B & C & D & E \\
\hline\hline 1 & 0 & 0 & 1 & 0 & 0 \\
\hline 2 & 1 & 0 & 0 & 0 & 0 \\
\hline 3 & 0 & 1 & 0 & 0 & 0 \\
\hline 4 & 0 & 1 & 0 & 0 & 0 \\
\hline 5 & 1 & 0 & 0 & 0 & 0 \\
\hline 6 & 0 & 1 & 0 & 0 & 0 \\
\hline 7 & 0 & 0 & 0 & 0 & 1 \\
\hline 8 & 0 & 1 & 0 & 0 & 0 \\
\hline 9 & 0 & 0 & 0 & 0 & 1 \\
\hline 10 & 0 & 0 & 0 & 1 & 0 \\
\hline 
\end{tabular}
\end{center}
\end{table}

\subsection{Implementation in Essence}
Implementation of Self Referential Quiz can be found in figure \ref{srq-essence-implementation}.

\lstloadlanguages{C++}

\definecolor{bgcolor}{rgb}{1,1,0.6} % definice barvy

\lstset{         % nastaveni prostredi listings
  language=C++,
	basicstyle=\tiny,
  keywordstyle=\bfseries,
  identifierstyle=\itshape,
  stringstyle=\ttfamily,
  numbers=none,
  numberstyle=\tiny,
  numbersep=5pt,
  frameround=ffff,
  extendedchars=true,
  tab=~,
  tabsize=2,
  frame=single,
  captionpos=b,
  showspaces=false,
  showstringspaces=false,
  breaklines=true
}


\begin{figure}[ht]
\caption{\label{srq-essence-implementation}Implementation of Self Referential Quiz in Essence}
\begin{lstlisting}
language ESSENCE' 1.b.a
find s : matrix indexed by [int(1..10), int(1..5)] of bool
such that
$ the is only one answer to each question and there is not any unanswered question
 forall row : int(1..10). ((sum col : int(1..5). s[row,col]) = 1),
$ Question 1
$ A to D
  forall col : int(1..4). ( (s[1,col] = 1) <=> ( (s[(4-col+1),1] = 1) /\ ( forall row : int(1..(4-col)). (s[row,1] = 0) ) ) ),
$ E
  (s[1,5] = 1) <=> (forall row : int(1..4). (s[row,1] = 0)),

$ Question 2
  forall col : int(1..5). ( (s[2,col] = 1) <=> ( forall col2: int(1..5). (s[3+col-1,col2] = s[3+col,col2]) ) ),
$ Question 3
  forall col : int(1..5). ( (s[3,col] = 1) <=> ( (s[(4+col-1),1] = 1) /\ ( forall row : int (4..2+col). s[row,1] = 0 ) ) ),
$ Question 4
  forall col : int(1..5). ( (s[4,col] = 1) <=> ( (s[col*2,2] = 1) /\ ( forall row : int(1..(col-1)). s[row*2,2] = 0 ) ) ),
$ Question 5
  forall col : int(1..5). ( (s[5,col] = 1) <=> (s[2*col-1,3]=1) ),
$ Question 6
  (s[6,1] = 1) <=> ( ( exists row : int(1..5). s[row,4] = 1 ) /\ ( forall row : int (7..10). s[row,4] = 0 ) ),
  (s[6,2] = 1) <=> ( ( exists row : int(7..10). s[row,4] = 1 ) /\ ( forall row : int (1..5). s[row,4] = 0 ) ),
  (s[6,3] = 1) <=> ( ( exists row : int(7..10). s[row,4] = 1 ) /\ ( exists row : int (1..5). s[row,4] = 1 ) ),
  (s[6,4] = 1) <=> ( forall row : int (1..10). s[row,4] = 0 ),
  (s[6,5] = 1) <=> (s[6,4] = 1),
$ Question 7
  forall col : int(1..5). ( (s[7,col] = 1) <=> ( (s[col+4,5] = 1) /\ ( forall row : int (col+4+1..10). s[row,5] = 0 ) ) ),
$ Question 8
  forall col: int(1..5). ( (s[8,col] = 1) <=> ( ( sum row: int(1..10). (s[row,2] + s[row,3] + s[row,4]) ) = (7-col+1) ) ),
$ Question 9
  forall col: int(1..5). ( (s[9,col] = 1) <=> ( ( sum row: int(1..10). (s[row,1] + s[row,5]) ) = (col-1) ) )
$ Constraints for question 10 are useless
\end{lstlisting} 
\end{figure}


\section{Quasigroup Completion Problem}
A Quasigroup or Latin square is an $n \times n$ table filled with n different symbols in such a way that each symbol occurs exactly once in each row and exactly once in each column. In Quasigroup Completion Problem (QCP) the task is to complete filling of given partially filled Latin square. Problem with QCP is inequality of difficulty of various initial settings. Filling of Latin square is NP-complete problem which means that for large squares the execution time can be enormous. Therefore we are unable to determine if solver cannot find solution or if there is not any. Even in case that given initial settings can be solved we do not have guarantee that all settings with the same percentage of filling are equally hard. Achlioptas et al. \cite{Achlioptas00generatingsatisfiable} defined a backbone. The backbone of instance is set of fields which are similarly evaluated in all solutions (not counting preassigned fields). There exists correlation between size of backbone and hardness of the given problem. We expect that if the backbone is near 0\% there exist lots of solutions and solver can find some "accidentally". In the opposite if the backbone is near 100\% all constraints leads to one solution. Therefore we expect interesting initial settings with backbone near 50\%. Experiments \cite{Achlioptas00generatingsatisfiable} proved that maximal computational time is with backbone between 30 and 35\%.

We will avoid the possibility of unsolvable settings by modification of given task. We will first generate complete quasigroup and then leave some of its fields empty. This will assure that given initial settings can be always solved. This modification is called Quasigroup with Holes (QWH) and QCP is its generalization. We will use given set of QWH initial settings.

\subsection{Formal definition}
Let have table $Q$ of size $n \times n$ whose fields can be filled with numbers in range $1$ to $n$ with following constraints:
\begin{enumerate}
	\item $\forall i,j,k \in 1..n : Q(i,j) \neq Q(i,k)$  [In each row all values are different]
	\item $\forall i,j,k \in 1..n : Q(i,j) \neq Q(k,j)$  [In each column all values are different]
	\item $Q(i,j)= x \Leftrightarrow$ The initial setting of cell $i$,$j$ $= x$.
\end{enumerate}

\subsection{Example solutions}

\subsection{Implementation in Essence}
\lstloadlanguages{C++}

\definecolor{bgcolor}{rgb}{1,1,0.6} % definice barvy

\lstset{         % nastaveni prostredi listings
  language=C++,
	basicstyle=\tiny,
  keywordstyle=\bfseries,
  identifierstyle=\itshape,
  stringstyle=\ttfamily,
  numbers=none,
  numberstyle=\tiny,
  numbersep=5pt,
  frameround=ffff,
  extendedchars=true,
  tab=~,
  tabsize=2,
  frame=single,
  captionpos=b,
  showspaces=false,
  showstringspaces=false,
  breaklines=true
}


\begin{figure}[ht]
\caption{Implementation of Quasigroup Completion Problem in Essence}
\begin{lstlisting}
language ESSENCE' 1.b.a
letting nDomain be domain int(0..n-1)
find qcp : matrix indexed by [nDomain,nDomain] of nDomain
such that
  forall i : nDomain. alldiff(qcp[i,0..n-1]),
  forall i : nDomain. alldiff(qcp[0..n-1,i])
\end{lstlisting} 
\end{figure}


\section{Locating warehouses}
We assume hypothetical company which wants to build warehouses which can supply existing stores with minimal costs. Each warehouse has defined number of stores which can supply. Also we know the costs of transporting of goods between planned warehouses and stores. The cost for building of each warehouse is the same. Our task is to select which warehouses have to be built such that all stores will be supplied and the total cost will be minimal. Compared to other benchmarks this benchmark is different. Locating warehouses is optimization task. For other benchmarks all solutions are equally good but here we can compare the solutions and the expected result is the best of them. This benchmark and its implementation in Mozart/Oz is described in \cite{mozart:documentation}.

\subsection{Formal definition}
Let have $n$ stores $S$ and $m$ warehouses $W$. For each store $s$ and each warehouse $w$ we have defined supply cost $SC(s,w)$. Building cost for every warehouse is $c$. We have to find the minimal evaluation of the objective function 

$$ \text{total cost} = \sum_{w \text{ is open}}\left(c + \sum_{s \text{ is suplied by } w}{SC(s,w)} \right) $$ 

