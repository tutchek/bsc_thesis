\chapter{Benchmark results}
\label{results}
\thispagestyle{myheadings}\markright{$ $Id$ $}

In this chapter we present the results of the performance measuring of each solver.
There are two types of tests. The robustness test and the performance test. In the robustness
test we compare the solvers according to the size of the task which the solver
was able to compute in a given time limit. In the performance tests we measure the time 
which is needed to find a solution to the problem and the consumed memory. We used 
for this purpose designed tool. The tool runs the solver executable with given parameters.
While the solver process is running the benchmarking tool periodically checks the \texttt{/proc/(pid)/stat} file
and saves the current state. If the elapsed time is larger than the given limit the 
\texttt{SIGINT} signal is sent to the process. The proces have one second to end
and produce some output. If the process is still running one second after the 
\texttt{SIGINT} signal the \texttt{SIGKILL} signal is sent and process is killed. 
In the robustness test the tool tries to estimate the size of the tast using the
binary search.

All of the benchmarks were performed on a single dedicated computer Pentium IV, 3GHz (single core 
with hyper threading),
1GB RAM running the Debian linux with linux kernel 2.6.18. Except the Mozart/Oz 
solver and SICStus Prolog all solvers were compiled on the target machine. The 
precompiled binaries distribution was used for the Mozart/Oz solver because the
source distribution contained in the time of writing of this thesis error which
prevented compiling of the code. The SICStus Prolog is not distributed as source 
code but only as a binary. Therefore we used the shipped executable. Since the 
SICStus Prolog is not a free open-source software we used a trial version of the solver.
There should be limited only the period when the solver can be used. The performance
of the solver should not be affected.

The following versions of the solvers were used:
\begin{itemize}
  \item Mozart 1.4.0-20080704
  \item Choco 2.1.0
  \item Minion 0.8.1
  \item Gecode 3.1.0
  \item \eclipse 6.0\_96
  \item SISCtus Prolog 4.0.7 -- linux, glibc 2.7
\end{itemize}   

\section{The robustness test}
In the robustness test we measure how long magic sequence the solver is able to compute in 
ten minutes. As stated in previous paragraph the benchmarking tool uses for this 
purpose the binary search. The tool have an interval $[a,b]$ of values on which the search 
is performed. It picks a value $v$ in the middle of the interval and tries to compute
the solution of that length. If it success it changes the interval to $[v+1,b]$, if 
it fails it changes the interval to $[a,v-1]$. Then the search continues in the same way
until the lower bound of the interval is larger than upper bound. The results of
the robustness test are in the table \ref{results:robustness}.

\begin{table}
\caption{\label{results:robustness}The results of the robustness test.}
\begin{center}
\begin{tabular}{cccccc}
\hline Mozart/Oz & Choco & Minion & Gecode & \eclipse & SICStus Prolog \\
\hline
  94 & 111 & 236 & 27 & 55 & 174 \\
\hline 
 
\end{tabular}
\end{center}
\end{table}

\section{The performance test}
This test compares the solvers according to the two citerias. The first criterium is
the time needed to solve given problem instances. The second criteria is the highest
peak of used computer memory during the computation. The following benchmark instances 
were used:

\begin{itemize}
  \item {\em queens10} -- 10 queens, find all solutions
  \item {\em queens100} -- 100 queens, find one solution
  \item {\em magic20} -- Magic sequence of the length 20, find one solution
  \item {\em srq} -- Self Refferential Quiz, find one solution
  \item {\em qwh20}
  \item {\em warehouses} -- Locating Warehouses, find the best solution with the initial setting as in the figure \ref{warehouses-setting} 
\end{itemize}  

\begin{table}
\caption{\label{results:performance:runtime}The results of the performance test. Runtime (ms)}
\begin{center}
\begin{tabular}{lrrrrrr}
\hline Benchmark & Moz & Cho & Min & Gec & ECL & SiP \\
\hline queens10 & 237.95 & 850.27 & 227.69 & 65.35 & 809.45 & 327.53 \\
	queens100 & 184.19 & 1005.02 & $\times$ & 34.70 & $\times$ & 15.40 \\
	magic20 & 11.48 & 1092.95 & 71.24 & 2926.86 & 4514.14 & 198.32 \\
	srq & & 482.81 & & 25.89 & 691.13 & 13.54 \\
	warehouses &  & 1500.83 & & 169.71 & 1808.29 & 14.41 \\  
\hline 
\end{tabular}
\end{center}
\end{table}

\begin{table}
\caption{\label{results:performance:runtime}The results of the performance test. Memory peak (MB)}
\begin{center}
\begin{tabular}{lrrrrrr}
\hline Benchmark & Moz & Cho & Min & Gec & ECL & SiP \\
\hline 	queens10 & 8.62 & 209.42 & 44.17 & 8.58 & 1158.15 & 38.08 \\
				queens100 & 23.31 & 209.46 & $\times$ & 6.71 & $\times$ & 1.35 \\
				magic20 & 2.32 & 1173.19 & 44.57 & 6.70 & 1157.99 & 39.30 \\
				srq & & 209.35 & & 6.68 & 1158.08 & 0.94 \\
				warehouses &  & 209.12 & & 7.00 & 1157.99 & 0.89 \\
\hline
\end{tabular}
\end{center}
\end{table}
