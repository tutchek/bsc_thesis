\section{Minion}

Minion is a constraint solver which works as a standalone application. Minion takes
as an input the problem description and returns the solutions if there are any.
Minion is open source and available at sourceforge.net. The problem description
is contained in a file with a special format. The file contains definitions of variables,
their domains and constraints on them. The best description of the input format could
be "the constraint assembly language". The constraints cannot use as their parameters 
expressions. If we need to constrain the expression we have to introduce a new variable
to be equal to the expression and then constrain this variable. For example let us assume 
we can use constraints $X = |Y|$ and $X < Y$ and we need to use constraint $X < |Y|$.
We have to introduce the auxiliary variable $a$ and post two constraints $a = |Y|$ and
 $X < a$. If we use many expressions in our problem it leads to enormous number of
 auxiliary variables. Moreover the language does not contain loops. Therefore we
 a cannot post various number of constraints based on the size parameter of the model.
 We have to state all constraints for a given problem instance. As a result the definition
 of 4-queens problem cannot be easily transformed to 8-queens problem. The set of constraints
 is limited and cannot be extended. Moreover some basic constraints have to be used in unusual way.
 If we want to constrain $a = \sum{w_i x_i}$ we have to use a pair of constraints
 $a \leq \sum{w_i x_i}$, $a \geq \sum{w_i x_i}$ since the variant with equal sign
 is not in the system. The used variable can be boolean, bounded integers, discrete integers
 and sparse bounded integers. Bool variable is a variable with domain $\{0,1\}$.
 Bounded integer variable is an integer variable where the solver stores only its bounds.
 Discrete integers are generally the same as bounded integers except they can contain 
 holes in the interval. Finally sparse bounded integers are variables which have a small
 number of non consecutive values specified in the file. During the computation
 only bounds of the spare bounded integers are updated.  
 
The input format is not very human-friendly. Even for small problems the number of
auxiliary variables grows over reasonable limits and the input is not easy to understand.
It is a reasonable solution to use a generator to generate a Minion input file.
User can either write his own ad hoc generator for his problem or use a tool like
Tailor which is described in the section \ref{tailor}.

\subsection{Solver description}
As stated in the previous section Minion is a standalone executable which takes as a parameter
filename of the input file or through the standard input the contents of input.
It automaticaly starts computation and writes on the standard output or into the
specified file solutions. It can handle only integers and therefore the problems have to
be encoded in a such way. The format of the input file is not stable but uses an identifier
which prevets from misinterpretation of the input. If the user feeds the Minion with 
an unsupported format of the file the solver informs about that and ends. The current version
of the input file is 3, and the user can recognize it by observing the first line of the input, which
has to contain only string \texttt{MINION 3}. After identification line there follows the sections
ot the file. The file can contain the following sections:

\begin{itemize}
  \item Definition of the variables
  \item Definition of the tupples
  \item Definition of the constraints
  \item Definition of the search parameter
\end{itemize}

The list of supported constraints can be found in the appendix \ref{constraints}.
The search engine behavior can be adjusted either in the input file or at the command line
by parameters. The user can adjust the order of variables during the selection phase of search,
decide whether to find all solutions or only one or how to format the output.

In the section \ref{implementation:minion} we included the implementation of the Magic
Sequence benchmark of the lenght five. As the reader can see the input file
is divided into several sections. The specification of the file format comes on the first line.
Then follows the sections of the file denoted by the label \texttt{*** SECTION NAME ***}. These
sections can be in the file in any order and repeated as many times as needed. The
only exception is the \texttt{EOF} section which denotes the end of the file and should be in
the file only once at the end of the file. We used the alternative model of the 
problem as described in the section \ref{benchmarks:ms}. Therefore we need five 
variables for the sequence and twenty-five auxiliary variables. We define them in 
the \texttt{VARIABLES} section on lines 4 -- 30. Then the constraints are defined in the 
\texttt{CONSTRAINTS} section. The implication is modelled using the reification constraint.
Since the Minion have only the constraints $\sum{x} \leq a$ and $\sum{x} \geq a$ but
 does not have the constraint $\sum{x} = a$ we have to use the constraints \texttt{sumleq} and
 \texttt{sumgeq}. Finally the \texttt{search} section defines the order of the variables.
 The variables will be searched in that order. The \texttt{SEARCH} section defines also the
 solution using the \texttt{PRINT} command. All variables marked for printing will
 be printed out as a solution. To generate the Minion input files we used the Tailor tool which
 compiled the Essence source code into the Minion input file; however, the large
 model were problematic for this tool. Therefore we had to write for some benchmarks
 the ad-hoc generators of the input file.
 
\subsection{Debugging support}
Minion offers printing out the search tree where one can observe the solvers actions.
However other tool than this printout is not available. Since the modelling in Minion
is difficult the user will probably use the Tailor tool. Tailor offers not only the translation
to the Minion format but direct invocation of the Minion, passing the input to it and
show the solution. The user then can directly modify the Essence source code and perform
the invokation again if the results are not what he expected. 

\subsection{Subjective description}
The largest problem of the solver is the limited set of constraints. If the problem needs
more sophisticated constraints then we cannot represent it in the Minion.
The second large problem is the input format. The problem can be resolved 
by using the Tailor tool; however, for the large problems the tool is not sufficient.
However as a standalone executable which can be called from the command line it can bring the capability of 
the constraint solving to programs written in the languages like bash. The only
thing which is needed is the capability to assemble the input file in the proper format. 
There is available the reference guide \cite{minion:manual} with description of all constraints available and
description of the input format.