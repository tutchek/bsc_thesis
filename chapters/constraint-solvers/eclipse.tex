\section{\eclipse}
\verb= $Id$ =

The \eclipse Constraint Programming System is an open source implementation of 
Prolog programming language. \eclipse is provided with libraries for constraint 
solving. It is possible to solve models over integers, real numbers and finite 
integer sets. The solver is not just one library but there is one general solver 
and several specialised solvers. Basic solver is the {\em ic} library -- interval 
constraints which contains basic arithmetic constraints. For global constraints there 
is a {\em ic\_global} library. Global constraints are constraints
which use some more advanced techniques to filter the variable domain, for example 
the \texttt{alldifferent} constraint implemented as a matching in a bipartite graph.
The ic library contains the alldifferent constraint too, but the one introduced in
ic\_global library is stronger. For scheduling problems there is an {\em ic\_cumulative}
library and for finite integer sets {\em ic\_sets} library. User is able to define his own
constraints if the shipped set of constraints is not satisfactory.

\subsection{Solver description}
As stated in previous paragraph \eclipse is an implementation of the Prolog programming language.
The system contains the solver as an independent library. Therefore the user is not 
limited to use only shipped solvers but can use his own solvers if there are any.
Standard solver is the {\em ic} solver which is hybrid finite domain and real number 
interval constraint solver. As the name suggest it offers constraining the variables with
both real and integer domains. Ic supports the arithmetic expressions, arithmetic constraints,
global constraints, reified constraints and search algorithms. The set of global constraints
can be extended by using {\em ic\_global} constraints. This library provides constraints
on the lists like \texttt{alldifferent}, \texttt{ordered}, \texttt{occurences} and so on.
For scheduling the mentioned {\em ic\_cumulative} constraint is available as well as 
their stronger versions {\em ic\_edge\_finder} respectively {\em ic\_edge\_finder3} which
differs in the time complexity of used algoriths. For the constraints over the symbolic 
domains like days of the week the {\em ic\_symbolic} domain is available.  The 
constraining over finite integer domains is also available using the {\em fd\_sets} library.
This library provides member constraints, cardinaity constraints, relation constraints
as well as set expressions. Problem is typically modelled as a prolog predicate which constrains the variables domains
then applies constraints on these variables and finally it calls a search algorithm 
on the variables. 

Capability of the solver can be extended by defining user defined constraints. For this
purpose the system is equiped with {\em Propia} library and the {\em Constraint Handling Rules}
library or {\em CHR}. CHR is a high level language for describing constraint rules. Reader can find
the description of CHR in \cite{fru_chr_book_2009}.
\eclipse offers {\em chr} library which can load source code in CHR format, translate it
into prolog predicate and then include it. The constraint then can be used as any standard
constrant shipped with the system.
The second way to introduce new constraint is using the Propia system. Propia takes 
any prolog predicate and convert it to the proper constraint.
The calling convence is \texttt{Goal infers most}. The library infers as many information about {\em Goal} 
as possible based on loaded constraint solver libraries. The level of infering can be adjusted.
Propia offers approximate generalised propagation. The {\em most} infering can be expensive 
to compute and maybe not necessary. The alternatives are predicates \texttt{Goal infers ic}, 
\texttt{Goal infers unique} and \texttt{Goal infers consistent}. As the name suggests, the {\em unique} 
infer ensures that all answers to the query are unique. The {\em consistent} infer can give
answer if the querzy can be solved or not. If it can be solved it additionaly checks, if the
constraint is already true. The {\em ic} infer is the same as {\em most} infer except that
{\em most} is based on currently loaded solvers opposed to {\em ic} which uses the specified solver.
We will show the example of using of our reimplemented $X \#> Y$ constraint which 
enforces that $X$ is greater than $Y$ using the standard $>$ operator. The problem predicate is the same 
as if we used only shipped constraints. However the $>$ operator cannot handle constrained
variables. And even if it could, it is not what we need. The standard behavior is to 
determine if $X$ is greater than $Y$ in the time of calling of the operator predicate and
that is all. We need that the operator keeps track on the variables and enforce the constraints
as the domain updates due to other constraints propagation. This is the point when the propia
comes. We mark the predicate as \texttt{infers most} and since that it is normal
constraint just like any other.

\begin{verbatim}
problem(X, Y) :-
  X :: [-100..100], 
  Y :: [-100..100], 
  (X > Y) infers most,  % predicate sent to Propia   
  X #= 5, 
  labeling([X,Y]).
\end{verbatim}

\subsection{Debugging support}
\eclipse offers visualisation related libraries, whose predicates can be applied 
on the variables. First the user has to create a viewable object using \texttt{create_viewable}
from {\em viewable} library which contains variables in proper order and so on. Then
user has to invoke a visualisation client, which is responsible for visualisation of the 
given viewable objects. The visualisation can be performed using  visualisation client
in the {\em java_vc} library. The created {\em viewable} objects are shown in
the visualisation client using several types of viewlets. and so on 
