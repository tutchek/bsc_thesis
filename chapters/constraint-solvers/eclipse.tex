\section{\eclipse}
\verb= $Id$ =

The \eclipse Constraint Programming System is an open source implementation of 
the Prolog programming language. \eclipse is provided with libraries for constraint 
solving. It is possible to solve models over integers, real numbers and finite 
integer sets. The solver is not just one library but there is one general solver 
and several specialised solvers. The basic solver is the {\em ic} library -- interval 
constraints which contains basic arithmetic constraints. For global constraints there 
is a {\em ic\_global} library. The global constraints are constraints
which use a couple of more advanced techniques to filter the variable domain, for example 
the \texttt{alldifferent} constraint implemented as a matching in a bipartite graph.
The ic library contains the alldifferent constraint too, but the one introduced in
ic\_global library is stronger. For scheduling problems there is an {\em ic\_cumulative}
library and for finite integer sets an {\em ic\_sets} library. A user is able to define their own
constraints if the shipped set of constraints is not satisfactory.

\subsection{Solver description}
As stated in the previous paragraph \eclipse is an implementation of the Prolog programming language.
The system contains the solver as an independent library. Therefore the user is not 
limited to use only shipped solvers, but they can use their own solvers if any.
A standard solver is the {\em ic} solver which is a hybrid finite domain and real number 
interval constraint solver. As the name suggests, it offers constraining the variables with
both real and integer domains. It supports arithmetic expressions, arithmetic constraints,
global constraints, reified constraints and search algorithms. The set of global constraints
can be extended by using {\em ic\_global} constraints. This library provides constraints
for lists such as \texttt{alldifferent}, \texttt{ordered}, \texttt{occurences} and so on.
For scheduling the mentioned {\em ic\_cumulative} constraint is available as well as 
their stronger versions {\em ic\_edge\_finder} or {\em ic\_edge\_finder3}, both of which
vary in the time complexity of used algoriths. For the constraints over the symbolic 
domains -- such as days of the week -- the {\em ic\_symbolic} domain is available.  The 
constraining over finite integer domains is also available using the {\em fd\_sets} library.
This library provides member constraints, cardinality constraints, relation constraints,
as well as set expressions. The problem is typically modelled as a prolog predicate which constrains the variables domains,
then applies constraints on these variables and, finally, it calls a search algorithm 
on the variables. 

A capability of the solver can be extended by defining user defined constraints. For this
purpose the system is equiped with {\em Propia} library and the {\em Constraint Handling Rules}
library or {\em CHR}. CHR is a high level language for describing constraint rules. The reader can find
the description of CHR in \cite{fru_chr_book_2009}.
\eclipse offers {\em chr} library which can load a source code in CHR format, translate it
into prolog predicate and then include it. The constraint then can be used as any standard
constraint shipped with the system.
The second way to introduce new constraint is to use the Propia system. Propia takes 
any prolog predicate and convert it to the proper constraint.
The calling convence is \texttt{Goal infers most}. The library infers as many information about {\em Goal} 
as possible based on the loaded constraint solver libraries. The level of infering can be adjusted.
Propia offers an approximate generalised propagation. The {\em most} infering can be expensive 
to compute and may not be necessary. The alternatives are predicates \texttt{Goal infers ic}, 
\texttt{Goal infers unique} and \texttt{Goal infers consistent}. As the name suggests, the {\em unique} 
infer ensures that all answers to the query are unique. The {\em consistent} infer can give
answer if the querzy can be solved or not. If it can be solved, it additionally checks whether the
constraint is already true or not. The {\em ic} infer is the same as the {\em most} infer except that
the {\em most} is based on currently loaded solvers compared to {\em ic} which uses the specified solver.
We shall show the example of using of our reimplemented $X \#> Y$ constraint which 
enforces that $X$ is greater than $Y$ using the standard $>$ operator. The problem predicate is the same 
as if we used only shipped constraints; however, the $>$ operator cannot handle constrained
variables. And even if it could, it is not what we need. The standard behavior is to 
determine if $X$ is greater than $Y$ in the time of calling of the operator predicate and
that is all. We need the operator to keep track on the variables and enforce the constraints
as the domain updates due to the other constraints propagation. This is the point when the Propia
comes. Once we mark the predicate as \texttt{infers most}, it is normal
constraint just like any other.

\begin{verbatim}
problem(X, Y) :-
  X :: [-100..100], 
  Y :: [-100..100], 
  (X > Y) infers most,  % predicate sent to Propia   
  X #= 5, 
  labeling([X,Y]).
\end{verbatim}

\subsection{Debugging support}
\eclipse offers visualisation related libraries, which predicates can be applied 
on the variables. First the user has to create a viewable object using \texttt{create\_viewable}
from {\em viewable} library which contains variables in a proper order and so on. Then
user has to invoke a visualisation client, which is responsible for visualisation of the 
given viewable objects. The visualisation can be performed using a visualisation client
in the {\em java\_vc} library. The created {\em viewable} objects are shown in
the visualisation client using several types of viewlets and so on. 

\subsection{Benchmarks implementation}

\subsection{Subjective description}