\section{Choco}

Choco is a constraint solver which is implemented as a library in Java programming language.
It is distributed as a JAR package having a Javadoc documentation included. It is quite easy
to install it even for a beginners in Java and it lasts about five minutes in commonly used
IDEs. Since the Java is used, the Choco solver is available for various platforms
and operating systems. As far as it is not our goal to describe possibilities of the host environment, 
we are not about to further discuss the Java features. Choco is being developed
at Ecole des Mines de Nantes in France and it is freely available for download from 
SourceForge server. The main number of the current version is 2. Choco divides the problem solution
into two parts -- a model and a solver itself. The model contains variables and constraints given 
in the problem. Afterwards, the solver is given the model as an input and it tries to find a solution. 
Variables in the model can be integers, real numbers or sets. Then the solver is able to
find a solution for the current model. A user can get information from the solver whether the 
problem has a solution or it contains a conflict. There is an interface for resolving solutions themselves, 
whereby one can ask for the first, the following or all existing solutions. If we define a variable
equal to a value of an objective function, the solver can either minimize or maximize this variable .
Furthermore, the solver allow us to choose a strategy which might perfectly fit the given problem.
The variables of the solver depends on the variables of the model and one can resolve the values only through
the variables of the solver.

\subsection{Solver description}
As it has been already mentioned in the previous section, the ptoblem solving is divided
into two separated tasks -- to define a model and to deploy the model to a well-configured solver.
The model as well as the solver are the independent Java objects. First we describe the 
model and after that we look at the solver. 

\subsubsection{Model}
The model is a instance of class CPModel. In Choco the variables of the model are represented
as objects of the following types: IntegerVariable, RealVariable and SetVariable. Those 
variables, generally, are not created using the keyword new, but in Choco there are
factory methods for this purpose. One has to register those variables first by calling
a function \texttt{CPModel::addVariable}, or \texttt{CPModel::addVariables} when adding an array of variables at once.
While registering variables into a model, we can set additional properties to the variables, 
for example to set whether it is a decision variable, or a variable containing a result
of an objective function. It is not necessary to set those properties every time; however,
they might rapidly improve the computation time. Alternatively, we can define those properties
in the solver, which way is described later in the following section.

Once we have registered the variables, a definition of constraints follows. It is possible
either to use a large number of build-in constraints, or to define our own constraints.
The constraints which are available natively in the solver are listed in the appendix \ref{constraints}.
Each constraint fits in one of the following groups: basic constraints (true, false, relation operators),
basic expressions (goniometric functions, powers, sums), other constraints (\texttt{abs}, \texttt{div}, \texttt{max}, ...),
reified contraints (\texttt{and}, \texttt{or}, \texttt{ifOnlyIf}) and global constraints (\texttt{allDifferent}, 
\texttt{occurenceMax}, ...). Furthermore, there are constraints available which might be used for modeling geometric constraints,
scheduling constraints and constraints for a sequence of variables which is accepted by finite automaton.

Apart from the build-in constraints, is it possible to define own constraints. The first step is to define 
a constraint $p(x,y)$ as a set of compatible values $(a,b)$, where $p$ is satisfied
if $x=a$ and $y=b$, or, eventually, as a set of incompatible values. In that case the
set is defined as a table of value. Besides, we can define the constraint as a predicate, which has to
be satisfied, whereby the constraint is an instance of a class derived from a class BinRelation
with a method checkCouple having implemented. This function takes two values as params and 
returns boolean value whether the condition was satisfied or not. Simirarly, we can define 
constraints over tuples. For all such constraints (either binary or tuple) we can specify the 
desired algorithm for arc consistence. There is AC3, AC2001, AC3rm and AC3 available for binary constraints and
 AC32, AC3rm, AC2001 and AC2008 tuple constraints. These algorithms are the variants of
 the original AC algorithm. Further description for some of them can be found in \cite{bartak:ogcp}.
 
As we find out while implementing of the Self Refferential Quiz the \texttt{ifOnlyIf} constraint
is implemented as an extensional constraint (the table of compatible values is computed).
If the the constrained variable has a large domain then the table is larger than
the possible memory.

\subsubsection{Solver}
A solver is an instance of class CPSolver, which tries to find a solution according to the model
from the previous section. The solver starts with reading the variables of the model and 
converting them into variables of the solver (IntegerVariable into IntDomainVar, RealVariable into RealVar and SetVariable into SetVar).
Afterwards, it reads the constraints of the model and creates constraints of the solver, which are based on previously read constraints.
Then the solver uses a search strategy and searches for solutions. Since the chosen strategy
is a key factor for the speed of solving, one can configure its various options.
A user can specify a selector and an iterator. Where the selector specifies which variable
is about to be taken in the next solver's decision and the iterator chooses each of available values
and iterates over them. In a standard distribution of Choco there are basic selectors such as 
{\em variable with minimum domain}, {\em variable with maximum domain} and so on. 
The iterators can try values in ascendant or descendant order. An alternative to iterator
is a value selector, which returns next available value when required. As for value selector,
we can use, for example, the minimal value in a domain, a random value in a domain and so on. We can choose
different user-defined strategies for various groups of variables so as to follow the specified problem
in the best way. In that case we define the solver's behaviour through so-called goals. A goal
contains a definition of a strategy, that means a selector for certain variables and an iterator over values.

Solving large-scale problems might be enormously time demanding, take too much system resources and so on.
To avoid this we can define solver limits. In the solver we can set a time limit, a limit for a number of nodes, 
a depth of backtracking, a number of fails or a limit for CPU time. Apart from that, a user can define their own limits. 

Once the solver has read the model and the strategies are defined, it starts solving the problem.
The solver offers an interface for accessing either each solutions (solve, nextSolution), or to get
all the solutions at once. Moreover, we can specify a variable which the solver tries to minimize or maximize.
Since the result is held in variables of the solver and not in user-defined variables of the model,
it is required to resolve the solver's variables by calling a function CPSolver::getVar, which accepts a variable of a model
and returns a variable of a solver.

\subsection{Debugging support}
Choco does not include any tools for the graphic visualisation of the search tree such as the systems Mozart or Gecode do;
however, in Choco it is possible to print out a log of the solving process.
One can configure several levels how detailed information is logged varying from
nothing to a complete list of what the Choco does internally.

\subsection{Subjective description}
The system has a good documentation \cite{choco:documentation}, although it is a little bit bad organised. Even though the careful readers
find virtually everything they look for. A documentation for developers is generated
by JavaDoc system. Due to that fact it is available as a hinting tool for many users of common Java IDEs, 
that definetely helps for better understanding of the solver. Since the development of the solver is
maintained at SourceForge server, it is quite easy to access source codes as well as a history of versions via
revision control system Subversion. One can find there also a technical support forum, where the authors
answer the users' questions. The reaction time is very low and the answers are of high
quality so most of the problems are quickly fixed.