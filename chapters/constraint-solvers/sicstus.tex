\section{SICStus Prolog}

SICStus Prolog is an implementation of the Prolog programming language by Swedish
Institute of Computer Science or SICS. It differs from the other here discussed solvers 
because SICStus Prolog is not a free open source system. The trial version of this
system can be obtained. Just like the \eclipse system the SICStus Prolog is not only
a constraint solver but a programming language where constraint solver is shipped as 
an independent library. The system provides libraries for solving the constraint problems
over the finite domains, over the boolean domains and over the real domains.

\subsection{Solver description}
Similarly as in the \eclipse system the solvers are independent libraries shipped 
with the system. The constraint solver over the finite domain consist of the
{\em clp(fd)} library \cite{Carlsson97anopen-ended}. The language constructs are different
compared to \eclipse; however, the basic properties are preserved. Therefore porting
the \eclipse program to SICStus Prolog is not a hard work but also it is not trivial.
The number of available constraints is higher than the \eclipse system provides.
The clp(fd)library offers standard arithmetic expressions, the relation constraints
and the global constraints. The library also provides the scheduling constraints and 
the extensional constraints.

The {\em clp(b)} provides a solver over the boolean variables. The solver contains
the relation constraints as well as constraints for the tautology and satisfiability.
Finally the {\em clp(q,r)} library \cite{clpqr:opai} provides constraining over the 
rational and real numbers. The constraint solver for set variables is not available.

The standard set of constraints can be expanded by the user defined constraints. As well as the
\eclipse system the SICStus Prolog offers to add constraints using the Constraint Handling
Rules language and FlatZinc language. User can also define his own constraints as the
Prolog predicate. The user created constraint can be either a global constraint or
a primitive constraint. The ways how to create such constraints are described in 
the SICStus Prolog manual.

The constraint library clfpd is very similar to the \eclipse library ic. We used the 
benchmarks already implemented in the \eclipse system and changed a very small amount
of library-specific calls to get the SICStus Prolog source code.

\subsection{Debugging support}
The SICStus Prolog offers the {\em Finite Domain Constraint Debugger} or {\em fdbg} library.
This library can be used next to the standard Prolog debugging predicate \texttt{trace}. This
predicate allows to debug any Prolog program. The {\em fdbg} library provides the
log of the constraint propagation and distribution. There are logged all changes to the
domains and other events which occurred during the computation. The example of the {\em fdbg}
output is in the figure \ref{sicstus:fdbg}. On the first line there is a Prolog query.
We constrain the variable $X$ to be in the domain $\{5,...,9\}$ and the variable $Y$
to be in the domain $\{1,...,6\}$. Then we constrain the variable $X$ to be smaller than
variable $Y$. We did not perform the search so the variables remains not solved.
User can set names for the variables to achieve a better orientation in the printout.
The library uses the following events which results in the action. The {\em constraint 
event} which is invoked when the global constraint is woken and the {\em labelling event}
which is called after the variable labeling is started or a variable gets constrained
or the labeling has failed. After such an event occurs the {\em visualizer} is called. 
The visualizer is a Prolog predicate which typically shows the event in the user 
friendly format; however, it can do any action instead as a response to the event.
The primitive constraints are not tracked by {\em fdbg} but the arithmetical constraints
like $X \#< Y$ are changed to the global constraints when the library is loaded.
The user can specify the filename where the output of the {\em fdbg} will be written.

\begin{figure}
\caption{\label{sicstus:fdbg}The example of the {\em fdbg} output.}
\begin{verbatim}
| ?- X in 5..9, Y in 1..6, X #< Y.
<fdvar_9> in 5..9
    fdvar_9 = inf..sup -> 5..9
    Constraint exited.

<fdvar_10> in 1..6
    fdvar_10 = inf..sup -> 1..6
    Constraint exited.

<fdvar_9>+1#=<<fdvar_10>
    fdvar_9 = 5..9 -> {5}
    fdvar_10 = 1..6 -> {6}
    Constraint exited.
    
X in 4..6,
Y in 5..7 
\end{verbatim}   
\end{figure}

 
\subsection{Subjective description}
The SICStus Prolog is a professional solution. The variety of available libraries 
is really large; however if user need to solve constraint problem over the set variables 
it is useless. System is well documented and the manual is exhaustive \cite{sicstus:manual}. The system
is not available freely for the public use but the time limited trial version can
be acquired.