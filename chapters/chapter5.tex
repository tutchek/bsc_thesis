\chapter{Conclusions}
\thispagestyle{myheadings}\markright{$ $Id$ $}

\section{Benchmark results}

*** NEJPRVA SPOCITAT BENCHMARKY ***

\section{Which solver choose?}

It is hard question. First of all user has to decide, if problem which is to be 
solved can be represented in finite domain. If it is not possible user cannot use
Mozart and Minion systems. Other criterium is time efficiency. As we can see from
benchmarks *** DOPLNIT CO VIDIME Z BENCHMARKU ***. Since all systems runs on Linux 
and Windows operating systems user is not limited by operating system which is using.

If there is no need for fast computing of solutions and rather is needed user comfort
and ways how to achieve the most accurate model describing problem, the best answer
is to let user choose. Mozart system provides various tools to inspect solutions 
and user really gets very quick idea what it is the system doing inside. However Mozart comes
with its own programming language which has to be learned to produce good results.
Using Emacs as default IDE can also be problem in start of using the solver (author
of this thesis is vim user so first steps with Mozart were tough). If user is familiar
with Prolog then \eclipse or SICStus Prolog is good solution for him. 



Another aspect is support by developers, community and documentation. Author of 
thesis has positive experience with Choco system, where authors answered in eight 
minutes to question posted to support forum. All systems have support forums and 
mailing lists where user can find answers to his questions.      
