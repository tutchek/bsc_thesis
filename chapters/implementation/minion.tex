\section{Minion}
\label{implementation:minion}

The Minion input file with the description of the magic sequence benchmark. 

\begin{lstlisting}[language=C++]
MINION 3

**VARIABLES**
DISCRETE s[5] {0..5}

BOOL aux0
BOOL aux1
BOOL aux2
BOOL aux3
BOOL aux4
BOOL aux5
BOOL aux6
BOOL aux7
BOOL aux8
BOOL aux9
BOOL aux10
BOOL aux11
BOOL aux12
BOOL aux13
BOOL aux14
BOOL aux15
BOOL aux16
BOOL aux17
BOOL aux18
BOOL aux19
BOOL aux20
BOOL aux21
BOOL aux22
BOOL aux23
BOOL aux24

**SEARCH**

PRINT[s]

VARORDER [
s , aux0, aux1, aux2, aux3, aux4, aux5, aux6, aux7, aux8, aux9, aux10, aux11, aux12, aux13, aux14, aux15, aux16, aux17, aux18, aux19, aux20, aux21, aux22, aux23, aux24]


**CONSTRAINTS**

reify(eq(0, s[0]), aux0)
reify(eq(0, s[1]), aux1)
reify(eq(0, s[2]), aux2)
reify(eq(0, s[3]), aux3)
reify(eq(0, s[4]), aux4)
sumleq([aux0,aux1,aux2,aux3,aux4], s[0])
sumgeq([aux0,aux1,aux2,aux3,aux4], s[0])
reify(eq(1, s[0]), aux5)
reify(eq(1, s[1]), aux6)
reify(eq(1, s[2]), aux7)
reify(eq(1, s[3]), aux8)
reify(eq(1, s[4]), aux9)
sumleq([aux5,aux6,aux7,aux8,aux9], s[1])
sumgeq([aux5,aux6,aux7,aux8,aux9], s[1])
reify(eq(2, s[0]), aux10)
reify(eq(2, s[1]), aux11)
reify(eq(2, s[2]), aux12)
reify(eq(2, s[3]), aux13)
reify(eq(2, s[4]), aux14)
sumleq([aux10,aux11,aux12,aux13,aux14], s[2])
sumgeq([aux10,aux11,aux12,aux13,aux14], s[2])
reify(eq(3, s[0]), aux15)
reify(eq(3, s[1]), aux16)
reify(eq(3, s[2]), aux17)
reify(eq(3, s[3]), aux18)
reify(eq(3, s[4]), aux19)
sumleq([aux15,aux16,aux17,aux18,aux19], s[3])
sumgeq([aux15,aux16,aux17,aux18,aux19], s[3])
reify(eq(4, s[0]), aux20)
reify(eq(4, s[1]), aux21)
reify(eq(4, s[2]), aux22)
reify(eq(4, s[3]), aux23)
reify(eq(4, s[4]), aux24)
sumleq([aux20,aux21,aux22,aux23,aux24], s[4])
sumgeq([aux20,aux21,aux22,aux23,aux24], s[4])

**EOF**
\end{lstlisting}
